\documentclass[10 pt]{article}
% 
\usepackage{amssymb,amsmath,amsthm,amscd}
%\usepackage[all]{xy}
\usepackage[latin1]{inputenc}
\usepackage{lipsum}
\usepackage[brazil]{babel}
%\usepackage{pdfsync} % sync the pdf file to inverse 'search' (using Sumatra)
%\usepackage[pdftex]{graphicx}
\usepackage{graphicx, enumerate}
\usepackage{tabularx}
\usepackage[table]{xcolor}
\usepackage{chngcntr}
\usepackage[autostyle]{csquotes}
\usepackage{hyperref}% to use \url{..} or \hyperref[label_name]{''link text''}
\renewcommand{\labelenumi}{\alph{enumi})}
\usepackage[autostyle]{csquotes}

\DeclareGraphicsRule{.wmf}{bmp}{}{}% declare WMF filename extension

%\usepackage{psfrag}

\addtolength{\textwidth}{100pt} \addtolength{\textheight}{58pt}
\addtolength{\hoffset}{-50pt} \addtolength{\voffset}{-15pt}



%%%%%%%%%%%%%%%%%%%%%%%%%%%%%%%%%%%%%%%%%%%%%%%%%%%%%%%%%%%%%%%%%%%%%%%%
\providecommand{\U}[1]{\protect\rule{.1in}{.1in}}
%EndMSIPreambleData
\setlength{\topmargin}{-0.9 in} \setlength{\textwidth}{6.50 in}
%\setlength{\oddsidemargin}{0.2 in} \setlength{\evensidemargin}{0.2 in}
\setlength{\textheight}{9.5 in} \setlength{\marginparwidth}{0.5 in}
\setlength{\marginparsep}{0.5 in}
\renewcommand{\baselinestretch}{1.5}


\newcommand{\comment}[1]{\marginpar{\quad\sffamily{\tiny #1 \par}}}
%\usepackage[notref,notcite]{showkeys}

%\usepackage[hypertex]{hyperref}
%\usepackage{hyperref}


\numberwithin{equation}{section}

\counterwithin{table}{section}

% theorems, some fancy :)

\newcommand{\sol}{\noindent{\bf Solu��o: }}
\newtheoremstyle{fancy1}{10pt}{10pt}{\itshape}{12pt}{\textsc\bgroup}{.\egroup}{8pt}{}

\newtheoremstyle{fancy2}{10pt}{10pt}{}{0pt}{\itshape}{:}{8pt}{ }

\newtheoremstyle{EP}{10pt}{7pt}{}{1pt}{\textsl{\textbf{Quest�o}} }{: }{}{#2}

\theoremstyle{fancy2}
\newtheorem{q}{{\textsl{\bf Quest�o}}}


\theoremstyle{fancy1}
\newtheorem{cor}[equation]{Corollary}
\newtheorem{lem}[equation]{Lemma}
\newtheorem{prop}[equation]{Proposition}
\newtheorem{thm}{Teorema}
\newtheorem{problem}{Problem}
\newtheorem{main}{Theorem}
\newtheorem*{main*}{Theorem}
\newtheorem*{conj}{Conjecture}
\newtheorem*{cor*}{Corollary}
\newtheorem{claim}{Claim}
\newtheorem*{mainc}{Corollary C}


\newtheorem*{problem*}{Problem}
\newtheorem{exercise}{Exercise}

%\renewcommand{\themain}{\Alph{main}}
%\renewcommand{\thetable}{\theequation}
%\setcounter{table}{\value{section}}



\theoremstyle{fancy2}
\newtheorem{definition}{Defini��o}
\newtheorem{rem}{Observa��o}
\newtheorem*{rem*}{Remark}
\newtheorem{exemplo}{Exemplo}



%%%%%%%%%%%%%%%%%%%%%%%%%%%%%%%%%%%%%%%%%%%%%%%%%%%%%%%%%%%%%%%%%%%%%%%%


\newcommand{\aref}[1]{Addendum~\ref{#1}}
\newcommand{\cref}[1]{Corollary~\ref{#1}}
\newcommand{\dref}[1]{Definition~\ref{#1}}
\newcommand{\eref}[1]{Example~\ref{#1}}
\newcommand{\lref}[1]{Lemma~\ref{#1}}
\newcommand{\pref}[1]{Proposition~\ref{#1}}
\newcommand{\rref}[1]{Remark~\ref{#1}}
\newcommand{\tref}[1]{Theorem~\ref{#1}}
\newcommand{\sref}[1]{Se��o~\ref{#1}}


%%%%%%%%%%%%%%%%%%%%%%%%%%%%%%%%%%%%%%%%%%%%%%%%%


%Greek letters

\newcommand{\ga}{\alpha}
\newcommand{\gb}{\beta}
\newcommand{\gc}{\gamma}
\newcommand{\gd}{\delta}
\newcommand{\gl}{\lambda}
%\newcommand{\gL}{\Lambda}
\newcommand{\gt}{\theta}
\newcommand{\gep}{\epsilon}
%\newcommand{\gS}{\Sigma}
\newcommand{\gs}{\sigma}
\newcommand{\eps}{\varepsilon}
\newcommand{\e}{\epsilon}
\newcommand{\gC}{\Gamma}
\newcommand{\gS}{\Sigma}




%%%%%%%%%%%%%%%%%%%%%%%%%%%%%%%%%%%%%%%%%%%%%%%%%


%complex proj. space, quat. proj. space,  Cayley plane, Sphere


\newcommand{\RP}{\mathbb{R\mkern1mu P}}
\newcommand{\CP}{\mathbb{C\mkern1mu P}}
\newcommand{\HP}{\mathbb{H\mkern1mu P}}
\newcommand{\CaP}{\mathrm{Ca}\mathbb{\mkern1mu P}^2}
%\newcommand{�}{\mathbb{S}}
\newcommand{\Sph}{\mathbb{S}}


%%%%%%%%%%%%%%%%%%%%%%%%%%%%%%%%%%%%%%%%%%%%%%%%%


% complex, real, integers

\newcommand{\C}{{\mathbb{C}}}
\newcommand{\R}{{\mathbb{R}}}
\newcommand{\Z}{{\mathbb{Z}}}
 \newcommand{\Q}{{\mathbb{Q}}}
\newcommand{\N}{{\mathbb{N}}}
\newcommand{\QH}{{\mathbb{H}}}
%\newcommand{\Ra}{{\mathbb{Q}} }



%%%%%%%%%%%%%%%%%%%%%%%%%%%%%%%%%%%%%%%%%%%%%%%%%


% Lie groups

% \renewcommand{\H }{\ensuremath{\operatorname{H}}}
% \newcommand{\I}{\ensuremath{\operatorname{I}}}
% \newcommand{\E}{\ensuremath{\operatorname{E}}}
% \newcommand{\F}{\ensuremath{\operatorname{F}}}
% \newcommand{\G}{\ensuremath{\operatorname{G}}}
% \newcommand{\D}{\ensuremath{\operatorname{D}}}
% \newcommand{\SO}{\ensuremath{\operatorname{SO}}}
% 
% %\renewcommand{�}{\ensuremath{\operatorname{O}}}
% \newcommand{\Sp}{\ensuremath{\operatorname{Sp}}}
% %\newcommand{\U}{\ensuremath{\operatorname{U}}}
% \newcommand{\SU}{\ensuremath{\operatorname{SU}}}
% \newcommand{\Spin}{\ensuremath{\operatorname{Spin}}}
% \newcommand{\Pin}{\ensuremath{\operatorname{Pin}}}
% \newcommand{\T}{\ensuremath{\operatorname{T}}}
% \renewcommand{�}{\ensuremath{\operatorname{S}}}
% \renewcommand{�}{\ensuremath{\operatorname{P}}}
% \newcommand{\No}{\ensuremath{\operatorname{N}}}
% 
% \newcommand{\K}{\ensuremath{\operatorname{K}}}
% \renewcommand{\L}{\ensuremath{\operatorname{L}}}
% \newcommand{\W}{\ensuremath{\operatorname{W}}}
% %\newcommand{\Q}{\ensuremath{\operatorname{Q}}}

%%%%%%%%%%%%%%%%%%%%%%%%%%%%%%%%%%%%%%%%%%%%%%%%



%Lie algebras

\newcommand{\fg}{{\mathfrak{g}}}
\newcommand{\fk}{{\mathfrak{k}}}
\newcommand{\fh}{{\mathfrak{h}}}
\newcommand{\fm}{{\mathfrak{m}}}
\newcommand{\fn}{{\mathfrak{n}}}
\newcommand{\fa}{{\mathfrak{a}}}
\newcommand{\fb}{{\mathfrak{b}}}
\newcommand{\fr}{{\mathfrak{r}}}
\newcommand{\fp}{{\mathfrak{p}}}
\newcommand{\fgl}{{\mathfrak{gl}}}
\newcommand{\fl}{{\mathfrak{l}}}
\newcommand{\fso}{{\mathfrak{so}}}
\newcommand{\fsu}{{\mathfrak{su}}}
\newcommand{\fu}{{\mathfrak{u}}}
\newcommand{\ft}{{\mathfrak{t}}}
\newcommand{\fz}{{\mathfrak{z}}}
\newcommand{\fsp}{{\mathfrak{sp}}}
\newcommand{\ff}{{\mathfrak{f}}}


\newcommand{\fspin}{{\mathfrak{spin}}}
\newcommand{\fspi}{{\mathfrak{spin}}}

%%%%%%%%%%%%%%%%%%%%%%%%%%%%%%%%%%%%%%%%%%%%%%%%%


% inner products, mods, brackets

\newcommand{\pro}[2]{\langle #1 , #2 \rangle}
\newcommand{\gen}[1]{\langle #1 \rangle}
\newcommand{\orth}[2]{\text{S(O(#1)O(#2))}}
\def\con#1=#2(#3){#1 \equiv #2 \bmod{#3}}
\newcommand{\Lbr}{\bigl(}
\newcommand{\Rbr}{\bigr)}
\newcommand{\bml}{\bigl\langle}
\newcommand{\bmr}{\bigr\rangle}
\newcommand{\ml}{\langle}                     % Riemannian metric (left )
\newcommand{\mr}{\rangle}                    % Riemannian metric (right)

\newcommand{\pel}{\parallel}                    % parallel lines

% arrows


\def\jjoinrel{\mathrel{\mkern-4mu}}
\def\llongrightarrow{\relbar\jjoinrel\longrightarrow}
\def\lllongrightarrow{\relbar\jjoinrel\llongrightarrow}
\def\Lllongrightarrow{\relbar\jjoinrel\relbar\jjoinrel\llongrightarrow}

%\newcommand{\ra}{\rightarrow}
\newcommand{\hra}{\hookrightarrow}
\newcommand{\lra}{\longrightarrow}
\newcommand{\llra}{\longleftrightarrow}
\newcommand{\Lra}{\Longrightarrow}
\newcommand{\Ra}{\Rightarrow }
\newcommand{\La}{\Leftarrow }
%%%%%%%%%%%%%%%%%%%%%%%%%%%%%%%%%%%%%%%%%%%%%%%%%


% mathematical operators
\newcommand{\sign}{\ensuremath{\operatorname{sign}}}

\newcommand{\tr}{\ensuremath{\operatorname{tr}}}
\newcommand{\diag}{\ensuremath{\operatorname{diag}}}
\newcommand{\Aut}{\ensuremath{\operatorname{Aut}}}
\newcommand{\Int}{\ensuremath{\operatorname{Int}}}
\newcommand{\rank}{\ensuremath{\operatorname{rk}}}
%\newcommand{\dim}{\ensuremath{\operatorname{dim}}}
\newcommand{\codim}{\ensuremath{\operatorname{codim}}}
\newcommand{\corank}{\ensuremath{\operatorname{corank}}}
\renewcommand{\Im}{\ensuremath{\operatorname{Im}}}
\newcommand{\Ad}{\ensuremath{\operatorname{Ad}}}
\newcommand{\ad}{\ensuremath{\operatorname{ad}}}
\newcommand{\diam}{\ensuremath{\operatorname{diam}}}
\renewcommand{\sec}{\ensuremath{\operatorname{sec}}}
\newcommand{\Ric}{\ensuremath{\operatorname{Ric}}}
\newcommand{\ric}{\ensuremath{\operatorname{ric}}}
\newcommand{\vol}{\ensuremath{\operatorname{vol}}}
\newcommand{\II}{\ensuremath{\operatorname{II}}}

\DeclareMathOperator{\Sym}{Sym} \DeclareMathOperator{\Real}{Real}
\DeclareMathOperator{\Or}{O} \DeclareMathOperator{\Iso}{Iso}
\DeclareMathOperator{\symrank}{symrank}
\DeclareMathOperator{\symdeg}{symdeg}
\DeclareMathOperator{\cohom}{cohom}
\DeclareMathOperator{\kernel}{kernel}
\DeclareMathOperator{\Map}{Map} \DeclareMathOperator{\Fix}{Fix}
\DeclareMathOperator{\trace}{trace} \DeclareMathOperator{\sgn}{sgn}
\DeclareMathOperator{\arccotan}{arccotan}
\DeclareMathOperator{\parity}{par} \DeclareMathOperator{\pr}{pr}
\DeclareMathOperator{\ind}{index} \DeclareMathOperator{\Ker}{Ker}
\DeclareMathOperator{\Kernel}{Ker}
     \DeclareMathOperator{\Exp}{Exp}
\DeclareMathOperator{\Hom}{Hom} \DeclareMathOperator{\id}{id}
\DeclareMathOperator{\Id}{Id} \DeclareMathOperator{\im}{im}
\DeclareMathOperator{\Gr}{Gr} \DeclareMathOperator{\nul}{null}
\DeclareMathOperator{\spam}{span}


%%%%%%%%%%%%%%%%%%%%%%%%%%%%%%%%%%%%%%%%%%%%%%%%%


% tilde's hat's bar's

\newcommand{\wh}{\widehat}


%%%%%%%%%%%%%%%%%%%%%%%%%%%%%%%%%%%%%%%%%%%%%%%%%

% good looking + and +/-

\newcommand{\Kpmo}{K_{\scriptscriptstyle{0}}^{\scriptscriptstyle{�}}}
\newcommand{\Kpo}{K_{\scriptscriptstyle{0}}^{\scriptscriptstyle{+}}}
\newcommand{\Kmo}{K_{\scriptscriptstyle{0}}^{\scriptscriptstyle{-}}}
\newcommand{\Kpm}{K^{\scriptscriptstyle{�}}}
\newcommand{\Kp}{K^{\scriptscriptstyle{+}}}
\newcommand{\Km}{K^{\scriptscriptstyle{-}}}
\newcommand{\Ko}{K_{\scriptscriptstyle{0}}}

\newcommand{\Ho}{H_{\scriptscriptstyle{0}}}
\newcommand{\Ktpm}{\tilde{K}^{\scriptscriptstyle{�}}}
\newcommand{\Kbpm}{\bar{K}^{\scriptscriptstyle{�}}}
\newcommand{\Kbp}{\bar{K}^{\scriptscriptstyle{+}}}
\newcommand{\Kbm}{\bar{K}^{\scriptscriptstyle{-}}}
\newcommand{\Ktm}{\tilde{K}^{\scriptscriptstyle{-}}}
\newcommand{\Ktp}{\tilde{K}^{\scriptscriptstyle{+}}}
\newcommand{\Hb}{\bar{H}}
\newcommand{\Ht}{\tilde{H}}

\newcommand{\ovl}{\overline}





\newcommand{\Shp}{S^{\scriptscriptstyle{+}}}
\newcommand{\Shm}{S^{\scriptscriptstyle{-}}}
\newcommand{\Shpm}{S^{\scriptscriptstyle{�}}}
\newcommand{\Dp}{D^{\scriptscriptstyle{+}}}
\newcommand{\Dm}{D^{\scriptscriptstyle{-}}}
\newcommand{\Dpm}{D^{\scriptscriptstyle{�}}}
\newcommand{\Bp}{B^{\scriptscriptstyle{+}}}
\newcommand{\Bm}{B^{\scriptscriptstyle{-}}}
\newcommand{\Bpm}{B^{\scriptscriptstyle{�}}}
\newcommand{\lp}{l^{\scriptscriptstyle{+}}}
\newcommand{\lm}{l^{\scriptscriptstyle{-}}}
\newcommand{\lpm}{l^{\scriptscriptstyle{�}}}

%general small looking subscript 0

\newcommand{\subo}{_{\scriptscriptstyle{0}}}


%%%%%%%%%%%%%%%%%%%%%%%%%%%%%%%%%%%%%%%%%%%%%%%%%


% useful ones

\newcommand{\no}{\noindent}
\newcommand{\co}{{cohomogeneity}}
\newcommand{\coo}{{cohomogeneity one}}
\newcommand{\com}{{cohomogeneity one manifold}}
\newcommand{\coms}{{cohomogeneity one manifolds}}
\newcommand{\coa}{{cohomogeneity one action}}
\newcommand{\coas}{{cohomogeneity one actions}}

\newcommand{\ka}{K�hler }
\newcommand{\holo}{holomorphic }
\newcommand{\herm}{Hermitian}
\newcommand{\cpt}{compact }
\newcommand{\mfld}{manifold}
\newcommand{\hbc}{holomorphic bisectional curvature }
\newcommand{\sff}{second fundamental form}
\newcommand{\spa}{\mbox{span}}
\newcommand{\kk}{\kappa}

\renewcommand{\sc}{sectional curvature}
\newcommand{\cu}{curvature}
\newcommand{\nn}{non-negative}
\newcommand{\nnc}{non-negative curvature}
\newcommand{\nnsc}{non-negative sectional curvature}
\newcommand{\pc}{positive curvature}
\newcommand{\psc}{positive sectional curvature}
\newcommand{\pcu}{positively curved}
\newcommand{\qp}{quasi positive}
\newcommand{\ap}{almost positive}


\newcommand{\np}{nonprincipal}

% \input{unirioprojeto.tex}

\usepackage[brazil]{babel}
\usepackage[utf8]{inputenc}

\begin{document}

% \titulo{Unirio de portas abertas}{2.2010}
% \includegraphics[scale=0.25]{unirio.png} %\includegraphics[scale=0.7]{faperj}


\begin{center}
  {\large{\bf   Livro Aberto de Matemática}}
\end{center}


\section{Identificação do projeto}
\textbf{T\'itulo:} Livro Aberto de Matemática

\begin{flushright}
  \noindent
  \begin{tabular}{llll}
    &&\\
    &\textbf{Professor coordenador:}& Fábio Luiz Borges Simas (UNIRIO)\\
    &\textbf{Pesquisadores associados:}& Augusto Quadros Teixeira (IMPA)
  \end{tabular}
\end{flushright}
\vspace{0.3cm}


\section{Resumo}
Este é um esforço de professores da Educação Básica e do Ensino Superior para produzir, até o final de 2017, uma coleção de livros didáticos de Matemática com código aberto, para o segundo ciclo do Ensino Fundamental, nos moldes do Plano Nacional do Livro Didático (PNLD). A esta obra será atribuída a licença {\it Creative Commons} (CC)  BY 4.0, isto significa que o material poderá ser livremente distribuído e alterado, mesmo que para fins comerciais. A redação ficará inicialmente a cargo da equipe organizadora que utilizará adaptações de Trabalhos de Conclusão de Curso (TCC) de graduação e mestrado, também serão aproveitados trabalhos de autores que aceitem disponibilizá-los nos termos da licença desta coleção. Posteriormente, espera-se que outros professores colaborem com a aplicação dos recursos produzidos em suas aulas, com comentários e até mesmo editando o material. Esta cooperação se dará através de um site explicando a filosofia do projeto, onde será possível visualizar, baixar e comentar o texto já produzido. Após cadastrar-se, o visitante também poderá editar o texto no próprio site, assim como na Wikipédia. Com a diferença de que neste projeto todas as edições serão avaliadas por uma equipe de revisão valendo-se da mesma tecnologia de controle de versões utilizada para {\it softwares} livres.


\section{Relevância da proposta}
Ao ser construída com a filosofia {\it open source} esta obra adquire um caráter permanente, de modo que independente dos organizadores, qualquer pessoa ou editora pode apropriar-se total ou parcialmente dos recursos nela disponíveis desde que respeitando os termos da licença %(\url{http://creativecommons.org/licenses/by/4.0/deed.pt_BR}).

O preço de venda dos livros desta coleção não pode ser excessivo porque qualquer um pode editar e imprimir o mesmo material disponível no site. Acredita-se que a ``Educação Aberta'' tenha potencial para readequar os preços dos livros didáticos brasileiros em um nível mais baixo.

Apesar de toda a beleza desta iniciativa, de pouco ela vale se os professores das escolas decidirem não usufruir dos recursos disponíveis em nossos livros. Neste sentido será fundamental a popularização e a divulgação massiva do material ainda em fase de elaboração entre os professores das escolas. Na medida em que ele se torna autor ou colaborador do material didático, torna-se natural que ele o escolha para trabalhar com seus alunos. Além disso, espera-se que o professor se coloque numa postura mais criativa no processo de ensino em contraposição ao lugar de mero reprodutor do conteúdo na ordem e do modo que são apresentados no livro didático escolhido.


\section{Introdução}
O livro didático na maioria das vezes é a principal ferramenta utilizada pelos professores da escola para prepararem as suas aulas. Contudo

A experiência dos programas Open Source.

Falar do OER Commons

Preço e qualidade dos livros didáticos aprovados no PNLD. (artigo do Nilson José Machado Em Aberto, Brasília, ano 16, n.69, jan./mar. 1996)

Escola e professores podem adaptar o texto para seus estudantes. Professor deve se sentir autor do material para que dele se aproprie. BNC e ``A forma básica de utilização [do livro didático], no entanto, foi mantida, e o livro 'adotado' pelo professor - consumível ou não - praticamente determina o conteúdo a ser ensinado. O professor abdica do privilégio de projetar os caminhos a serem trilhados, em consonância com as circunstâncias - experiências, interesses, perspectivas - de seus alunos, passando a conformar-se, mais ou menos acriticamente, com o encadeamento de temas propostos pelo autor. Tal encadeamento ora tem características idiossincráticas, ora resulta da cristalização de certos percursos, que de tanto serem repetidos, adquirem certa aparência de necessidade lógica; nos dois casos, a passividade do professor torna um pouco mais difícil a já complexa
tarefa da construção da autonomia intelectual dos alunos.'' (artigo do Nilson José Machado Em Aberto, Brasília, ano 16, n.69, jan./mar. 1996 página 31).


Acreditamos que a democratização da educação passa pelo Open Source.

Potencial para uso em outras disciplinas escolares.


% \section{Relevância da proposta}
% Valor da educação.

\section{Objetivos}

\begin{enumerate}
\item Produzir uma coleção de livros didáticos de matemática com código aberto e de livre distribuição e edição contendo quatro volumes para os estudantes e quatro manuais de professores nos moldes do PNLD.
\item Melhorar a qualidade dos livros didáticos de matemática utilizados nas escolas públicas do segundo ciclo do Ensino Fundamental.
\item Reduzir os preços das coleções de livros de matemática concorrentes no PNLD para o segundo ciclo do Ensino Fundamental.
\item Conseguir adesão a esta coleção junto com aos professores das Escolas da Educação Básica.
\end{enumerate}

\section{Pontos críticos para o sucesso da obra}
A qualidade e adesão desta coleção pelos professores dependerá principalmente de dois pontos: o comprometimento da equipe organizadora com a filosofia da coleção e do alcance que ela terá entre os professores no momento de sua elaboração.

\section{Método} \label{metodo}
O projeto será apresentado em um site na internet, onde poderão ser visualizados e baixados o sumário de cada livro e os capítulos da coleção em suas versões mais recentes. Haverá um espaço para comentários sobre o conteúdo de cada capítulo. O usuário poderá editar o texto diretamente, assim como na Wikipédia. Para isso deverá se registrar e concordar com os termos da licença. Então com um clique acessará um editor de texto com o código em {\it latex} do capítulo que está visualizando. Ao término da edição, o revisor é informado das alterações linha por linha. Estas modificações somente serão incorporadas ao texto disponível para visualização após a concordância do primeiro revisor.

\subsection{Dos recursos computacionais}
O aparato tecnológico necessário já existe, falta apenas a junção de ferramentas para que a cooperação seja o mais simples possível. Na seção do site em que figura a visualização de dado arquivo, haverá link para o código-fonte do arquivo no {\it GitHub} (servidor de arquivos com serviço de compartilhamento e controle de versões já utilizado para projetos como este, por exemplo, neste servidor é desenvolvido o Linux). O {\it GitHub} informa as alterações feitas pelo usuário aos revisores.

\subsection{Do conteúdo}
O conteúdo será aquele determinado pela Base Nacional Curricular (BNC) complementado com alguns tópicos extras a serem selecionados. A produção inicial ficará a cargo de TCC de graduação e do Profmat sob a orientação dos professores envolvidos, também serão incorporados alguns trabalhos já existentes de autores que aceitem disponibilizá-los nos termos da licença desta coleção. Além disso, qualquer pessoa que tenha interesse, pode  editar o texto, modificando, excluindo ou adicionando novos trechos.

\subsection{Da experimentação do material}


\subsection{Da revisão}
Trabalharemos com ao menos um revisor por ano e um revisor por área, segundo a divisão do BNC (geometria; grandezas e medidas; estatística e probabilidade; números e operações; álgebra e funções). Assim serão ao menos nove professores selecionados do Ensino Fundamental, Médio e Superior trabalhando como revisores.


\subsection{Dos direitos autorais e da licença}

{\it Creative Commons 4.0} %\url{http://creativecommons.org/licenses/by/4.0/deed.pt_BR}

\subsection{Da editoração}
Todo o trabalho de diagramação, impressão e editoração será realizado pelo IMPA.

\section{A equipe}\label{equipe}
\begin{table}[ht]
  \begin{center}
    \begin{tabular}{ll}
      Professores universitários & Professores da Educação Básica  \\

    \end{tabular}
  \end{center}
\end{table}

\noindent{\it Responsabilidades de cada membro na equipe.}




\begin{enumerate}%[A.]
\item Revisor de ano
  \begin{enumerate}%[(1)]
  \item Analisar a relevância para a Coleção do texto submetido.
  \item Analisar a consistência matemática do texto submetido.
  \item Corrigir o uso da norma culta.
  \item Efetuar a adequação da linguagem para o Ensino Fundamental.
  \end{enumerate}

\item Revisor de área
  \begin{enumerate}%[(1)]
  \item Analisar a relevância para a Coleção do texto submetido.
  \item Analisar a consistência matemática do texto submetido.
  \item Corrigir o uso da norma culta.
  \item Efetuar a adequação da linguagem para o Ensino Fundamental.
  \end{enumerate}

\end{enumerate}

\section{Cronograma de execução}


\section{Bibliografia relacionada ao projeto}


\begin{center}
  Rio de Janeiro, 04 de janeiro de 2016 \hspace{2cm} Fabio Luiz Borges Simas
\end{center}


% estilo da bibliografia
\bibliographystyle{apalike}
% chamando o arquivo refs.bib
% \bibliography{refs}




\end{document}
