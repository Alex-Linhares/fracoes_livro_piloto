\documentclass{book}
\usepackage[margin=.5in]{geometry}


\usepackage[brazil]{babel}
\usepackage{template_professor}

\usepackage{todonotes}

\usepackage{microtype}

\begin{document}

\todo{1}

\section{Observação}   Objetivos específicos: Levar o aluno a
\begin{itemize} %s
    \item       Diferenciar a partição da unidade em partes       ``quaisquer''       da partição da unidade em partes       ``iguais''      . A partição em partes iguais será chamada equipartição.
    \item       Reconhecer a necessidade de uma expressão verbal que identifique uma das partes iguais em uma equipartição da unidade.
    \item       Diferenciar       ``a partição da unidade em três partes quaisquer''       da       ``partição da unidade em três partes iguais''      .
    \item       Compreender as expressões       ``um terço de''       e       ``terça parte de''       como formas de identicar uma das partes da equipartição da unidade em três partes.
\end{itemize} %s



  Recomendações e sugestões para o desenvolvimento da atividade:
\begin{itemize} %s
    \item       Recomenda-se que a atividade seja desenvolvida em grupos de 3 a 5 alunos.
    \item       Busque conduzir a discussão nos grupos de modo que os estudantes percebam que, para que os irmãos recebam a mesma quantidade de chocolate, a partição proposta para a barra de chocolate deve ser em       ``partes iguais''      , no sentido de ganharem todos a mesma quantidade de chocolate, não necessariamente pedaços de mesma forma.
    \item       Na discussão, procure destacar que a referência à       ``partição em três partes iguais''       se dá (igualmente) a partir das expressões       ``um terço''       da barra de chocolate ou       ``a terça parte''       da barra de chocolate.
    \item       O item (c ) admite diversas soluções, algumas estão apresentadas como resposta. No entanto, algumas dessas respostas podem não aparecer naturalmente em sala de aula. Avalie a possibilidade de apresentar e explorar algumas dessas soluções (ou outras que queira) em sala de aula. Por exemplo, apresente uma dessas divisões aos alunos e peça-os que avaliem a equipartição, explicando sua decisão.
    \item       O item (d), provavelmente, pode não ser respondido corretamente pelos alunos. Se for o caso, as expressões       ``um terço de''       e       ``a terça parte de''       devem ser apresentadas.
    \item       Fique atento às falas dos alunos. Observe que os alunos podem representar e verbalizar as respostas de diferentes modos e que não há uma resposta única para a atividade. Por exemplo, alguns alunos podem precisar de mais tempo do que outros para usar a expressão       ``um terço''       no lugar de       ``partição em três partes iguais''       ou       ``divisão em três partes iguais''      . Ou ainda, observarem que há diferentes representações para a equipartição. Por exemplo,
\end{itemize} %s


  \begin{imagem*}[breakable]{}{}
    PÁGINA DE REPRODUÇÃO

    - FIGURA ARTÍSTICA -

    Imagens: a mesma barra de chocolate que aparece na figura correspondente à atividade, dividida exatamente como na imagem (instrução a seguir), e as partes separadas para serem reproduzidas isoladamente (como no exemplo).

        \includegraphics[width=120pt, keepaspectratio]{../../livro/media/cap1/secoes/chocolate_2.jpg}

        \includegraphics[width=120pt, keepaspectratio]{../../livro/media/cap1/secoes/chocolate_3.jpg}
  \end{imagem*}

\begin{itemize} %s
    \item       Esta atividade pode ser adaptada para alunos com deficiência de visão. Para isso, sugere-se confeccionar os modelos da barra de chocolate  inteira e repartida, que estão disponíveis para reprodução (      {\bf Inserir LINK de página para reprodução}      ), em três materiais diferentes. Por exemplo, papel comum e papéis com texturas diferentes, tecido ou material emborrachado.
\end{itemize} %s





\todo{2}

\section{Resposta}
\begin{enumerate} [\quad a)] %s
    \item       Este item não possui resposta correta, apenas respostas coerentes com a explicação do aluno. Por exemplo, um estudante pode dizer que sim e explicar que o irmão mais velho deve ficar com uma parte maior porque precisa de mais energia. Mas a resposta esperada é que a divisão não é justa porque as quantidades de chocolate são diferentes. Discuta com os alunos para que entendam a divisão correspondente à resposta esperada.
    \item       Não, eles receberão quantidades diferentes de chocolate, embora cada um receba um único pedaço do chocolate.
    \item       Respostas possíveis:
\end{enumerate} %s
  \begin{imagem*}[breakable]{}{}      - FIGURA ARTÍSTICA -    \mbox{} \newline              \includegraphics[width=360pt, keepaspectratio]{../../livro/media/cap1/secoes/licao1_atv1.png}    \mbox{} \newline      ilustração: Cambrainha   \end{imagem*}
\begin{enumerate} [\quad a)] %s
    \item       Cada parte é       ``um terço''       da barra ou a       ``terça parte''       da barra.
\end{enumerate} %s



\todo{2}

\section{Observação}

  Objetivos específicos: Levar o aluno a

\begin{itemize} %s
    \item       Perceber que a unidade (no caso, uma pizza) pode ser subdividida em uma quantidade igual de partes sem que essa divisão represente uma equipartição.
    \item       Distinguir uma equipartição dentre partições diversas.
    \item       Diferenciar       ``a divisão da unidade em quatro partes quaisquer''       da       ``divisão da unidade em quatro partes iguais''      .
    \item       Compreender as expressões       ``um quarto de''       e       ``quarta parte de''       como forma de identicar uma das partes da equipartição em 4 partes.
\end{itemize} %s


  Recomendações e sugestões para o desenvolvimento da atividade:

\begin{itemize} %s
    \item       Recomenda-se que a atividade seja desenvolvida em grupos de 3 a 5 alunos.
    \item       As diversas soluções apresentadas pelos diferentes grupos devem ser discutidas com a turma inteira.
    \item       É possível que os alunos utilizem expressões variadas para nomear as partes de pizzas em cada divisão. Por exemplo,       ``a maior quarta parte''      ,       ``a menor quarta parte''      ,       ``as quartas partes iguais entre si''      ,       ``a menor parte''      ,       ``a maior parte''      , dentre outras. É importante que a discussão conduza os alunos ao entendimento de que apenas as partes da equipartição podem ser chamadas de       ``quartos''       da pizza, as demais são simplesmente fatias ou pedaços, por exemplo.
    \item       Os alunos devem reconhecer que apenas uma das repartições propostas sugere a equipartição, respondendo assim a última questão proposta nesta atividade.
    \item       Na ilustração, as crianças dos grupos em que a pizza não está em fatias com iguais quantidades de pizza parecem contrariadas. Recomenda-se levantar a questão:       ``Por que será que elas estão parecendo zangadas?''
    \item       Essa atividade pode ser adaptada para alunos com deficiência visual. Para isso, sugere-se confeccionar os modelos das três pizzas repartidas, que estão disponíveis para reprodução (      {\bf Inserir LINK de página para reprodução}      ), em três materiais diferentes. Por exemplo, papel comum e papéis com texturas diferentes, tecido ou material emborrachado.
\end{itemize} %s


  \begin{imagem*}[breakable]{}{}      - FIGURA ARTÍSTICA
    \begin{nota*}[breakable]{}{}       NA PÁGINA PARA REPRODUCAO - INCLUIR a imagem das 3 pizzas repartidas: inteiras e com as respectivas partes isoladas.
    \end{nota*}

        \includegraphics[width=300pt, keepaspectratio]{../../livro/media/cap1/secoes/licao1_atv2.png}
    ilustração: Cambrainha

  \end{imagem*}


\todo{3}

\section{Resposta}
\begin{enumerate} [\quad I)] %d
    \item       Sim. Cada grupo repartiu sua pizza em quatro fatias.
    \item       Não, pois algumas fatias têm quantidades de pizza diferentes das outras.
    \item       Apenas no grupo 1 as 4 crianças receberam a mesma quantidade de pizza. Cada fatia da pizza do grupo 1 é       ``um quarto''       da pizza ou       ``a quarta parte''       da pizza. Diferentemente das demais pizzas.
\end{enumerate} %d



\todo{3}

\section{Observação}
  Objetivo específico:Levar o aluno a
\begin{itemize} %s
    \item       Abordar a equipartição em um modelo linear.
    \item       Reconhecer a quarta parte como a metade da metade.
\end{itemize} %s


  Recomendações e sugestões para o desenvolvimento da atividade:

\begin{itemize} %s
    \item       Recomenda-se que esta atividade seja desenvolvida em grupos de quatro alunos.
    \item       Cada grupo deve receber um pedaço de barbante de, aproximadamente, 1m e quatro enfeites (todos iguais).
    \item       Os quatro enfeites precisam ser confeccionados antes da realização da tarefa. Sugerem-se estrelas, cujos modelos estão disponíveis para reprodução       {\bf (Inserir LINK de página para reprodução)}      . No entanto, segundo a avaliação do professor, os enfeites podem ser outros, desde que sejam os 4 congruentes.
    \item       Como sugestão, se possível, solicitar aos alunos que confeccionem os enfeites, por exemplo, associando esta atividade com geometria, com a abordagem de grandezas e medidas, com a disciplina de artes ou envolvendo culturas artesanais populares.
    \item       A equipartição do barbante não deve ser obtida a partir da medida do barbante, mas por sucessivas dobras do barbante sobre ele mesmo, como ilustrado na resposta da atividade.
    \item       A manipulação e a dobra do barbante devem sustentar a discussão para a identificação da       ``metade da metade''       com a       ``quarta parte''       do barbante. Nesse caso, a identificação se dará pela sobreposição das partes.
\end{itemize} %s



\todo{4}

\section{Resposta}   Uma maneira de se cortar o barbante é dobrar ao meio e depois dobrar novamente ao meio, obtendo quatro partes iguais, como ilustrado na figura a seguir.
  \begin{imagem*}[breakable]{}{}     - FIGURA ARTÍSTICA - Sequência de duas imagens que ilustrem um barbante sendo dobrado sobre ele mesmo duas vezes sucessivas, até que se obtenha quartos. Por exemplo, como a imagem a seguir:
        \includegraphics[width=180pt, keepaspectratio]{../../livro/media/cap1/secoes/barbante_dobras.jpg}
  \end{imagem*}


\todo{4}

\section{Observação}

  Nesta etapa, espera-se que os alunos compreendam as frações como forma de expressar quantidades. O objetivo é que percebam seu papel para expressar quantidades em situações de equipartição da unidade. Assim, as frações podem ser utilizadas no dia a dia para identificar quantidades do mesmo modo que os números naturais, já conhecidos dos alunos. Por exemplo, como nas expressões:   ``dois ovos''  ,   ``duas xícaras de farinha''  ,   ``um terço de xícara de cacau''   e   ``meio litro de leite''  .

  Objetiva-se a expressão verbal e não a representação simbólica. Espera-se, assim, que os alunos apropriem-se das expressões verbais que identificam as frações unitárias (um meio, um terço, um quarto, ... , um nono e um décimo) antes de serem apresentados formalmente à simbologia matemática (que será objetivo da próxima lição).  A referência às frações unitárias com a expressão   ``um''   antes da identicação da parte, como, por exemplo, em   ``um terço''   e em   ``um sétimo''   é uma decisão pedagógica. Claro que é possível se referir a essas frações simplesmente por   ``terço''   e   ``sétimo''  , respectivamente. No entanto, nas próximas seções, pretende-se que as frações não unitárias, como   ``dois terços''   e   ``nove sétimos''  , por exemplo, sejam entendidas a partir da justaposição das frações unitárias correspondentes, o que é naturalmente amparado pela contagem. Nas expressões verbais relativas às frações unitárias, o   ``um''   antes da identificação da parte está associado à cotagem. Dessa formma, a compreensão das frações   ``um terço''   e   ``dois terços''   ou das frações   ``um sétimo''   e   ``nove sétimos''  , por exemplo, seguem a mesma construção lógica.


\newpage

\todo{6}

\section{Observação}
  Objetivo específico: Levar o aluno a:
\begin{itemize} %s
    \item       Reconhecer que, em uma equipartição, as partes podem não ter a mesma forma.
    \item       Identificar a equivalência entre as partes de uma equipartição a partir de sobreposição ou da comparação pelo reconhecimento da associação a uma mesma fração unitária (no caso, 1/4).
    \item       Reconhecer a quarta parte como a metade da metade.
\end{itemize} %s


  Recomendações e sugestões para o desenvolvimento da atividade:

\begin{itemize} %s
    \item       Recomenda-se que esta atividade seja desenvolvida em grupos de 3 a 5 alunos. Cada grupo deve receber as imagens dos oito retângulos, disponíveis para reprodução       {\bf (Inserir LINK de página para reprodução)}       e colorí-las, cada um com uma cor diferente das demais.
    \item       Em cada grupo, os alunos devem decidir qual (ou quais) das divisões propostas para os retângulos correspondem a uma partição em quartos. É importante observar que todos os retângulos estão divididos em quartos.
    \item       Conduza a discussão de modo a levar os alunos a reconhecer que, em uma equipartição, as partes não precisam ter a mesma forma.
    \item       Se necessário, o professor pode associar cada retângulo a um objeto concreto (por exemplo, uma barra de chocolate ou a um pedaço de bolo). No entanto, nesta atividade, espera-se que os alunos consigam lidar com a figura de um retângulo como representativa de uma unidade genérica.
    \item       Recomenda-se que os alunos recortem as partes de cada um dos retângulos para realizar a comparação por sobreposição. No entanto, essa estratégia não será suficiente para todos os 8 casos. Em alguns casos, a comparação se dará pela identificação da fração unitária correspondente a cada parte. Nesses casos, o aluno deve reconhecer que a quarta parte é equivalente à metade da metade. Por exemplo, como no caso seguir.
\end{itemize} %s


  \begin{imagem*}[breakable]{}{}     Incluir imagem como a exemplificada a seguir, inclusive com os comentários indicados
        \includegraphics[width=120pt, keepaspectratio]{../../livro/media/cap1/secoes/metade_da_metade.jpg}
  \end{imagem*}

\begin{itemize} %s
    \item       Segundo a avaliação do professor, a atividade pode ser realizada em duas etapas. Em um primeiro momento, os alunos recebem quatro das oito imagens, Página A, e realizam a atividade com essas imagens - cuja comparação se dá apenas pela sobreposição. Em seguida, recebem as outras quatro, para concluir a atividade. Para as figuras da Página B, será necessário reconhecer a quarta parte como a metade da metade. É importante que o professor, ao final das duas etapas, avalie as escolhas como um todo.
\end{itemize} %s


  \begin{imagem*}[breakable]{}{}
    \begin{nota*}[breakable]{}{}       - FIGURA GEOMÉTRICA - PÁGINA PARA REPRODUÇÃO
      os retângulos devem estar organizados em duas páginas conforme a ilustração.
                   \includegraphics[width=120pt, keepaspectratio]{../../livro/media/undefined/quartos_encarte_1.jpg}                     \includegraphics[width=120pt, keepaspectratio]{../../livro/media/cap1/secoes/quartos_encarte_2.jpg}
    \end{nota*}
  \end{imagem*}



\todo{6}

\section{Resposta}
\begin{enumerate} [\quad a)] %s
    \item       Todos os retângulos estão divididos em quartos.
    \item       Dois desenhos possíveis são:
\end{enumerate} %s


\newpage

\todo{8}

\section{Observação}
  Objetivo específico: Levar o aluno a:

\begin{itemize} %s
    \item       Identificar uma mesma fração unitária (no caso, a terça parte) em representações diversas, ou seja, em representações de unidades não necessariamente congruentes.
\end{itemize} %s



  Recomendações e sugestões para o desenvolvimento da atividade:
\begin{itemize} %s
    \item       Recomenda-se que esta atividade seja desenvolvida em grupos de 3 a 5 alunos.
    \item       Durante a discussão, os alunos devem ser estimulados a explicar as suas escolhas. A discussão sobre os motivos da identificação, ou não, de cada uma das representações à terça parte da unidade correspondente será fundamental para atingir o objetivo da atividade.
    \item       Os alunos devem reconhecer que, independente da unidade considerada, em uma equipartição em 3 partes, cada uma das partes é um terço (ou a terça parte) da unidade.
    \item       Aproveite as próprias palavras e os argumentos dos alunos para conduzi-los às conclusões esperadas.
    \item       Fique atento aos alunos que selecionarem as figuras que simplesmente possuem alguma associação com o número 3, não correspondendo a terços. Por exemplo, um aluno que associe a       {\bf Figura ZZ}       a terços pode ainda não ter compreendido a necessidade da equipartição para a identificação de um terço. Já o aluno que associa       {\bf Figura ZZ}       a terços pode estar simplesmente contando as partes em vermelho, sem que tenha reconhecido que a figura deveria estar dividida em 3 partes iguais e não em 5.
\end{itemize} %s

\newpage

\todo{9}

\section{Resposta}
A parte em vermelho representa um terço da figura nos itens C), D), E), F) e H).

\newpage

\todo{10}

\section{Resposta}  
  Possibilidades de resposta. Incluir também respostas que sejam reuniões de partes não justapostas.  
  
    \includegraphics[width=360pt, keepaspectratio]{../../livro/media/cap1/secoes/licao1_atv61_novo.png}  
  
  $\ $     
  
    \includegraphics[width=420pt, keepaspectratio]{../../livro/media/cap1/secoes/licao1_atv6_2_novo.png}  
  
  ilustração:Cambrainha  


\todo{10}

\section{Observação}  
  
  Objetivos específicos: Levar o aluno a:  
\begin{itemize} %s
    \item       Representar uma fração unitária a partir de uma unidade dada.  
    \item       Reconhecer (e obter) um quarto como a metade da metade e um oitavo como a metade de um quarto.
    \item       Comparar as frações unitárias metade, um quarto e um oitavo de um mesmo quadrado.
\end{itemize} %s
  
  
  Recomendações e sugestões para o desenvolvimento da atividade:  
\begin{itemize} %s
    \item       Esta é uma atividade que o aluno pode fazer individualmente. 
    \item       Não se espera que, nesta atividade, os alunos usem a medida para fazer a equipartição de maneira mais precisa. O objetivo é fazer a equipartição livremente e de forma coerente. Assim, por exemplo, podem ser aceitas como respostas:
\end{itemize} %s
  
  
    \includegraphics[width=90pt, keepaspectratio]{../../livro/media/cap1/secoes/quadrado-resposta-01.png}   e     \includegraphics[width=90pt, keepaspectratio]{../../livro/media/cap1/secoes/quadrado-resposta-02.png}  .  
  
  Já as representações a seguir sugerem que os alunos precisam revisar os conceitos exigidos para a solução da atividade:  
  
    \includegraphics[width=90pt, keepaspectratio]{../../livro/media/cap1/secoes/quadrado-resposta-03.png}  
  
\begin{itemize} %s
    \item       A representação da unidade se dá de forma genérica por um quadrado. 
    \item       Espera-se que os alunos reconheçam que para obter um quarto da unidade basta tomar a metade da metade. E que, para determinar um oitavo pode dividir um quarto ao meio.
    \item       Recomende que os alunos usem dobradura para identificar as frações pedidas. Assim, por exemplo, a fração       $\frac{1}{4}$       pode ser obtida por duas dobras do papel.    
    \item       Discuta com os estudantes que quanto maior o número de partes iguais em que se particiona o quadrado, menor fica cada uma das partes.
    \item       Procure apresentar e discutir com a turma mais do que uma solução para cada item.
    \item             {\bf As diferentes soluções apresentadas pelos alunos podem enriquecer a discussão}      . A comparação entre, por exemplo, a metade do quadrado proveniente da dobradura pela diagonal e o quarto do quadrado proveniente da dobradura a partir de linhas paralelas aos lados (como um sinal de       ``+''      ) pode não ser tão natural. Dificuldade semelhante pode ser observada na comparação entre esse mesmo quarto do quadrado e o oitavo do quadrado proveniente de uma sequência de dobraduras paralelas a um dos lados, determinando       ``faixas paralelas''      . Nesses casos, para executar a comparação, é necessário que os alunos reconheçam partes de formatos diferentes que correspondem a uma mesma fração do quadrado como iguais em quantidade. Assim, a comparação entre a metade do quadrado, obtida pela dobradura na diagonal, e o quarto do quadrado, obtido pela dobraduta       ``em sinal de +''      , pode ser amparada pelo reconhecimento de que a metade em questão é igual em quantidade à metade do quadrado obtida por uma única dobra paralela a um dos lados, que é o dobro do quarto do quadrado.  
\end{itemize} %s
  
  
  
  


\todo{11}

\section{Resposta}  
  
  Algumas soluções possíveis, convencionais e outras menos convencionais são:  
  
        - Metade:   \mbox{} \newline        \includegraphics[width=240pt, keepaspectratio]{../../livro/media/cap1/secoes/licao1_atv7_metade.png}  
        - Um quarto:  \mbox{} \newline        \includegraphics[width=240pt, keepaspectratio]{../../livro/media/cap1/secoes/licao1_atv7_quarto.png}  
        - Um oitavo:   \mbox{} \newline        \includegraphics[width=240pt, keepaspectratio]{../../livro/media/cap1/secoes/licao1_atv7_oitavo.png}  \mbox{} \newline    Incluir também as soluções abaixo.  \mbox{} \newline        \includegraphics[width=100pt, keepaspectratio]{../../livro/media/cap1/secoes/fracoes_unitarias_do_quadrado.jpg}  \mbox{} \newline    ilustração: Cambrainha  
        - Dentre as opções apresentadas, a maior fração do quadrado é metade.  
  
  
  


\todo{11}

\section{Observação}  
  
  Objetivos específicos: Levar o aluno a:  
\begin{itemize} %s
    \item       Representar uma fração unitária (no caso, um meio ou metade) a partir de uma unidade dada.  
    \item       Estabelecer representações diferentes para a mesma fração unitária e para uma mesma unidade.
\end{itemize} %s
  
  
  Recomendações e sugestões para o desenvolvimento da atividade:  
\begin{itemize} %s
    \item       Essa é uma atividade que o aluno pode fazer individualmente. 
    \item       Como na atividade anterior, não se espera que, nesta atividade, o aluno use a medida para fazer a equipartição de maneira mais precisa. O objetivo é que o aluno faça a equipartição livremente e de forma coerente. 
    \item       Incentive os alunos a usar dobradura para decidir sobre as diferentes formas de identificar metades na unidade apresentada.
\end{itemize} %s
  
  * Observe que a representação da unidade se dá de forma genérica, ainda em modelo contínuo, por uma figura não tradicional como retângulos e círculos, que é determinada pela justaposição de dois hexágonos regulares.  
\begin{itemize} %s
    \item       Procure apresentar e discutir com a turma mais do que uma solução para cada item.
\end{itemize} %s
  


\todo{12}

\section{Resposta}  
  
  Algumas das respostas possíveis para este problema são:  
    \includegraphics[width=240pt, keepaspectratio]{../../livro/media/cap1/secoes/hexagonos_licao_um.jpg}  
  


\todo{12}

\section{Observação}  
  Objetivos específicos: Levar o aluno a:  
\begin{itemize} %s
    \item       Reconhecer a metade de uma unidade pela reunião de partes menores e em partições diversas. 
    \item       Estabelecer representações diferentes para a mesma fração unitária para uma mesma unidade.
\end{itemize} %s
  
  
  Recomendações e sugestões para o desenvolvimento da atividade:  
\begin{itemize} %s
    \item       Esta é uma atividade que o aluno pode fazer individualmente.
    \item       Esta atividade pretende levar o aluno a perceber que a metade de uma unidade pode ser considerada e identificada mesmo sem que se tenha uma divisão em duas partes iguais. 
    \item       Como nas atividades anteriores, não se espera que o aluno use a medida para confirmar a metade da unidade. O objetivo é que o aluno identifique a representação da metade (ou não) por sobreposição e justaposição das partes, decompondo e recompondo a figura.
    \item       Cada aluno deve receber as imagens das figuras, disponíveis para reprodução (      {\bf Inserir LINK para a página para reprodução}      ) para que possa manipular como achar melhor e conduzir a sua decisão. 
    \item       Incentive os alunos a argumentar, justificando a sua decisão. Para isso, podem, por exemplo, se apoiar em dobraduras ou no recorte das partes da figura.
    \item       Procure apresentar e discutir com a turma mais do que uma solução para cada item.   
\end{itemize} %s
  
  
  \begin{imagem*}[breakable]{}{}         
    \begin{nota*}[breakable]{}{}       PÁGINA PARA REPRODUÇÃO - Na página para reprodução deve conter as figuras da imagem. Estas imagens devem ter as dimensões especificadas na figura do exercício.      
    \end{nota*}    
  \end{imagem*}  


\todo{12}

\section{Resposta}  
  As figuras que correspondem à metade da unidade são as de números 1, 2, 4, 5, 6, 8, 9, 11 e 12.  



\todo{12}

\section{Observação}     
  Objetivos específicos: Levar o aluno a   
\begin{itemize} %s
    \item       Conhecer e compreender as expressões correspondentes as frações unitárias com denominadores de 5 a 10.
    \item       Comparar frações da unidade através da representação visual de frações do círculo.
\end{itemize} %s
  
      
  Recomendações e sugestões para o desenvolvimento da atividade:  
\begin{itemize} %s
    \item       Esta atividade pode ser resolvida individualmente, mas é essencial que seja discutida com toda a turma.  
    \item       É provável que nem todos os alunos conheçam ou intuam as expressões correspondentes às frações propostas. Nesse caso, cabe ao professor apresentá-las e diferenciá-las.
    \item       Aproveite esta atividade para revisar e discutir o vocabulário que é objetivo nesta seção:       {\it unidade,}             {\it metade,}             {\it um meio,}             {\it um terço,}             {\it terça parte,}             {\it um quarto,}             {\it quarta parte,}             {\it um quinto,}             {\it quinta parte,}             {\it um sexto,}             {\it sexta parte,}             {\it um sétimo,}             {\it sétima parte,}             {\it um oitavo,}             {\it oitava parte,}             {\it um nono,}             {\it nona parte,}             {\it um décimo}       e       {\it décima parte}      .
\end{itemize} %s
  
  


\todo{13}

\section{Resposta}  
\begin{enumerate} [\quad a)] %s
    \item       A correspondência adequada é:      
\begin{enumerate} [\quad I)] %d
        \item           A esta afirmação corresponde a figura G).
        \item           A esta afirmação corresponde a figura D).
        \item           A esta afirmação corresponde a figura I).
        \item           A esta afirmação corresponde a figura B).
        \item           A esta afirmação corresponde a figura A).
        \item           A esta afirmação corresponde a figura F).
\end{enumerate} %d

    \item       As frações um sétimo, um oitavo, um nono e um décimo do círculo são menores que um sexto do círculo. Qualquer uma delas está correta.
    \item       As frações um meio, um terço, um quarto, um quinto, um sexto, um sétimo e um oitavo do círculo são maiores que um nono do círculo. Qualquer uma delas está correta.
    \item       As frações um sétimo e um oitavo do círculo são menores que um sexto e maiores que um nono do círculo.
\end{enumerate} %s
  


\todo{13}

\section{Observação}  
  
  Objetivos específicos: Levar o aluno a:  
\begin{itemize} %s
    \item       Distinguir frações unitárias a partir de representações em modelos diversos, baseados em equipatição ou não. 
    \item       Comparar frações unitárias a partir de representações em modelos diversos, baseados em equipatição ou não.
    \item       Estabelecer a comparação entre as frações       ``um meio''      ,       ``um quarto''       e       ``um décimo''      . 
    \item       Reconhecer e diferenciar a representação das frações       ``um meio''      ,       ``um quarto''       e       ``um décimo''       em modelos diversos, baseados em equipatição ou não.
    \item       Estabelecer a comparação entre as frações       ``um meio''      ,       ``um quarto''       e       ``um décimo''       
\end{itemize} %s
  
  
  Recomendações e sugestões para o desenvolvimento da atividade:  
\begin{itemize} %s
    \item       Esta é uma atividade que o aluno pode fazer individualmente.
    \item       Esta atividade pretende levar o aluno a perceber que a metade de uma unidade pode ser considerada e identificada mesmo sem que se tenha uma divisão em duas partes iguais. 
    \item       Como nas atividades anteriores, não se espera que os alunos usem a medida para confirmar a metade. O objetivo é que identifiquem a representação da metade (ou não) por sobreposição e justaposição dasa partes, decompondo e recompondo a figura.
    \item       Cada aluno deve receber as imagens das figuras, disponíveis para reprodução (      {\bf Inserir LINK de página para reprodução}      ) para que possa manipular como achar melhor e conduzir a sua decisão. 
    \item       Incentive os alunos a argumentar, justificando a sua decisão. Para isso, podem, por exemplo, se apoiar em dobraduras ou no recorte das partes da figura.
    \item       Procure apresentar e discutir com a turma mais do que uma solução para cada item   
\end{itemize} %s
  


\todo{14}

\section{Resposta}   (A) um meio,  (B) um décimo, (C) um quarto, (D) um quarto,  \mbox{} \newline   
  (E) um quarto, (F) um meio, (G) um quarto, (H) um décimo,  \mbox{} \newline   
  (I) um quarto, (J) um décimo, (L) um quarto, (M) um meio  






\end{document}