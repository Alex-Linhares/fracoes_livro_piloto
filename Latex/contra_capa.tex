\documentclass[a4paper,10pt]{article}
%\documentclass[a4paper,10pt]{scrartcl}

\usepackage[utf8]{inputenc}
\usepackage[brazil]{babel}
\usepackage{tikz}
\usetikzlibrary{positioning}

\RequirePackage{geometry}
\geometry{
textwidth=153mm,
paperwidth=196mm,
paperheight=280mm,
top=1in,
left=20mm,
footskip=.7in
}

\usepackage{hyperref}

\begin{document}




\thispagestyle{empty}

\vspace{2cm}
\begin{center}
Projeto: LIVRO ABERTO DE MATEMÁTICA

\url{umlivroaberto.com}
\vspace{2cm}

\begin{tabular}{p{.15\textwidth}p{.7\textwidth}}
Título: & Frações no Ensino Fundamental - Volume 1\\
Ano: & 2016 \\
Editora & Instituto Nacional de Matem\'atica Pura e Aplicada (IMPA-OS)\\
Apoio:& Olimp\'iada Brasileira de Matem\'atica das Escolas P\'ublicas (OBMEP)\\
\\
Coordenação: & Fabio Simas e Augusto Teixeira\\
\\
Autores: & Cydara Cavedon Ripoll, Fabio Luiz Borges Simas, Humberto José Bortolossi, Victor Augusto Giraldo, Wanderley Moura Rezende, Wellerson  da Silva Quintaneiro\\
\\
Colaboradores: & Ana Paula Pereira (CAp UFF), Andreza Gonçalves (estudante da UFF), Bruna Luiza Oliveira (estudante da UFF), Francisco Mattos (Colégio Pedro II), Helano Andrade (estudante da UNIRIO), João Carlos Cataldo (CAp UERJ e Colégio Santo Ignácio), Luiz Felipe Lins (Secretaria de Educação da Cidade do Rio de Janeiro), Michel Cambrainha (UNIRIO), Rodrigo Ferreira (estudante da UNIRIO), Tahyz Pinto (estudante da UFF) \\
\\
Ilustradores: & Luiz Fernando Alves Macedo,
Vitoria da Mota Souza,
Eduardo Filipe de Miranda Souto,
Caio Felipe da Silva Evangelista,
Gisela Alves de Souza,
Mauricio de Azevedo Neto,
Briza Aiki Matsumura,
Vinícius Marcondes de Paula Silva,
Wanessa Souza de Oliveira,
Maurício Menegatti Andrade,
Eduardo Filipe de Miranda Souto,
Livia Machado da Silveira Verly,
Caio Felipe da Silva Evangelista,
Lucas Hideo Maekawa,
Lucas Oliveira Machado de Sousa,
Kayky Zigart Carlos e
Israel Fialho Magalhães\\
\\
Capa: & Fabio Simas
\end{tabular}
\vspace{5cm}
 
\includegraphics[scale=1]{cc}

Após o dia $1^{\textrm{\underline{o}}}$ de setembro de 2026 esta obra passa a estar licenciada por CC-by-sa.

Algumas figuras podem possuir licença com mais direitos do que a vigente para todo o material.
\end{center}
\pagebreak

\end{document}
