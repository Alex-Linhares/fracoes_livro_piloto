
\setcounter{chapter}{4}
\chapter{Adição e subtração de frações }

\section{EXPLORANDO O ASSUNTO }

\setcounter{subsection}{0}
\subsection{Atividade}


Miguel e Alice estão participando de uma campanha da escola para coleta de óleo de cozinha. O objetivo é disponibilizar recipientes para que as pessoas depositem óleo. Depois esses recipientes serão destinados a empresas que usarão o óleo descartado para fazer sabão. Eles conseguiram diferentes recipientes e agora desejam saber qual tem maior capacidade.

\begin{imagem*}[breakable]{}{}  

FIGURA ARTÍSTICA

Para o ilustrador: Incluir imagem de dois recipientes. Um em formato cilíndrico e outro em formato de cone (como sugerido pelas imagens). Ambas as figuras devem ter áreas de base próximas, sendo que a altura da figura em formato de cone deve ser 1,5 vezes a altura da figura em formato cilíndrico.

\begin{tabular}{ccc}
\includegraphics[width=100pt, keepaspectratio]{../../livro/media//cap5/ativ/garrafa_alice.png} &\quad \quad&  \includegraphics[width=100pt, keepaspectratio]{../../livro/media//cap5/ativ/garrafa_miguel.png}\\
Recipiente 1: trazido pela Alice & &Recipiente 2: trazido pelo Miguel 
\end{tabular} 
\end{imagem*}

Eles tiveram a seguinte ideia: encheram os dois recipientes com água para depois verificarem onde havia mais água. Para isso, usaram um copo d'água como unidade de medida. 
\begin{itemize}
 \item O recipiente trazido por Alice foi enchido com 26 copos.
 \item O recipiente trazido por Miguel foi enchido com 40 copos.
\end{itemize}
Eles então observaram que a partir de {\bf uma unidade de medida comum} (nesse caso o copo), poderiam não só dizer qual recipiente tinha maior capacidade, mas também o quanto era maior e qual seria a capacidade dos dois recipientes juntos. 
Usando a ideia de medida de Miguel e Alice, isto é, tomando o copo como unidade de medida, responda:
  \begin{enumerate}[a)]
   \item Qual recipiente tem maior capacidade?
   \item Qual é a capacidade dos dois recipientes juntos?
   \item Quanto se deve retirar do recipiente maior, para ter o mesmo volume de líquido que é possível colocar no recipiente menor?
  \end{enumerate}


\subsection{Atividade}

A professora Estela quer enfeitar sua sala de aula para uma festa da escola. Para isso ela comprou várias fitas, todas de mesmo tamanho, nas cores vermelho, azul e amarelo.
  
\begin{center}
\begin{tikzpicture}[x=1.0cm,y=1.0cm, scale=.5]
\draw[fill=attention] (0.,1) rectangle (12.,3.); 
\draw[fill=common] (0.,-2) rectangle (12.,0.);
\draw[fill=yellow,fill opacity=.3] (0.,-5) rectangle (12.,-3.);
\end{tikzpicture}
\end{center}


A professora cortou cada fita vermelha em 3 partes iguais, cada fita azul em 2 partes iguais e cada fita amarela em 4 partes iguais.

\begin{center}
\begin{tikzpicture}[x=1.0cm,y=1.0cm, scale=.5]
\draw[fill=attention] (0.,1) rectangle (12.,3.); 
\foreach \x in {4,8} \draw[dashed] (\x,1) -- (\x,3);
\draw[fill=common] (0.,-2) rectangle (12.,0.);
\draw[dashed] (6,-2) -- (6,0);
\draw[fill=yellow,fill opacity=.3] (0.,-5) rectangle (12.,-3.);
\foreach \x in {3,6,9} \draw[dashed] (\x,-5) -- (\x,-3);
\end{tikzpicture}
\end{center}

\begin{enumerate} [\quad a)] %s
  \item     A que fração da fita original corresponde cada pedaço recortado pela professora Estela?    \mbox{} \newline           \mbox{} \newline      Em seguida, a professora Estela começou a juntar pedaços recortados das fitas, formando novas fitas coloridas. Ela começou juntando (de forma intercalada) um pedaço azul e dois pedaços amarelos.     \mbox{} \newline      

\begin{center}
\begin{tikzpicture}[x=1.0cm,y=1.0cm, scale=.5]
\draw[fill=yellow,fill opacity=.3] (0.,0) rectangle (3,2);
\draw[fill=common] (3,0) rectangle (9.,2.);
\draw[fill=yellow,fill opacity=.3] (9,0) rectangle (12,2);
\end{tikzpicture}
\end{center}

Ela verificou que a nova fita formada tinha o mesmo tamanho da fita original. Isso aconteceu por que cada pedaço azul tem o mesmo tamanho de dois pedaços amarelos. Podemos representar o tamanho da nova fita formada pela professora por meio de uma {\bf soma de frações}. Cada pedaço azul corresponde a $\frac{1}{2}$ da fita original. Cada pedaço amarelo corresponde a $\frac{1}{4}$ da fita original, então 2 pedaços amarelos correspondem a $\frac{2}{4}$ da fita original. Portanto, o tamanho da nova fita é igual a: $$\dfrac{1}{2} + \dfrac{2}{4}.$$ Mas, como $\frac{2}{4}$ é igual a $\frac{1}{2}$ (cada pedaço azul tem o mesmo tamanho de dois pedaços amarelos), então: $$\dfrac{1}{2} + \dfrac{2}{4} = \dfrac{1}{2} + \dfrac{1}{2}.$$ O resultado dessa soma $\frac{1}{2} + \frac{1}{2}$ é igual 2 pedaços de $\frac{1}{2}$, isto é, $\frac{2}{2}$ (que é igual 1). Assim: $$\dfrac{1}{2} + \dfrac{2}{4} = \dfrac{1}{2} + \dfrac{1}{2} = 1.$$ Neste caso, o resultado 1 corresponde ao tamanho da fita original.
  \item     A professora também agrupou pedaços de fita, juntando 1 pedaço amarelo e 1 pedaço azul, como na figura abaixo. Qual fração da fita inicial corresponde esses dois pedaços juntos?
\end{enumerate} %s

   
\begin{imagem*}[breakable]{}{}   FIGURA ARTÍSTICA - Inserir uma caixa com uma professora falando:   ``Um dos objetivos dessa lição é compreender como é juntar frações quaisquer. Observar, por exemplo, como fica a fração da fita composta por pedaços amarelo e vermelho, onde em que um não tem o dobro do tamanho do outro.''\end{imagem*}

\subsection{Atividade}


Uma barra de chocolate é vendida com as marcações mostradas na figura abaixo.
\begin{imagem*} [breakable]{}{} 
 figura artística
 
\includegraphics[width=100pt, keepaspectratio]{../../livro/media//cap5/ativ/chocolate_16_pedacos.png}
\end{imagem*}
 

Alice comeu a metade dessa barra de chocolate (em bege), quebrou o restante da barra em pedaços, seguindo as marcações e comeu 3 desses pedaços (em azul).

\begin{center}
\begin{tikzpicture}[x=1.0cm,y=1.0cm, scale=.7]
\fill[fill=light] (-1.,5.) -- (-1.,1.) -- (3.,1.) -- (3.,5.) -- cycle;
\fill[fill=common] (3.,5.) -- (5.,5.) -- (5.,3.) -- (3.,3.) -- cycle;
\fill[fill=common] (5.,5.) -- (7.,5.) -- (7.,4.) -- (5.,4.02) -- cycle;
\draw  (-1.,5.)-- (-1.,1.);
\draw  (-1.,1.)-- (3.,1.);
\draw  (3.,1.)-- (7.,1.);
\draw  (7.,1.)-- (7.,5.);
\draw  (7.,5.)-- (-1.,5.);
\draw  (3.,5.)-- (3.,3.);
\draw  (5.,5.)-- (5.,1.);
\draw  (-1.,3.)-- (7.,3.);
\draw  (-1.,2.)-- (7.,2.);
\draw  (1.,5.)-- (1.,1.);
\draw  (-1.,4.)-- (7.,4.);
\draw  (3.,3.)-- (3.,1.);
\end{tikzpicture}
\end{center}

Se considerarmos a barra de chocolate como a unidade indicamos que as quantidades comidas são: $\frac{1}{2}$ por Alice e $\frac{3}{16}$ por Miguel.
Os pedaços da barra (quebrados por Miguel de acordo com as marcações na barra) correspondem a uma subdivisão dessa unidade.
Observe que ambas as frações da barra de chocolate comidas por Alice e Miguel podem ser obtidas a partir dessa subdivisão: Miguel comeu 3 pedaços e a quantidade comida por Alice corresponde a 8 pedaços.
\begin{enumerate}[a)]
\item Um pedaço corresponde a que fração da barra de chocolate?
\item Complete a parte em branco (numerador) para indicar a fração da barra de chocolate que Alice comeu. $$\frac{1}{2} = \frac{ }{16}$$ 
\item Que fração da barra de chocolate foi comida por Alice e por Miguel, juntos?
\item  Que fração da barra de chocolate não foi comida?
\end{enumerate}

\subsection{Atividade}


Amanda, Bruno e Caio pediram três pizzas do mesmo tamanho, mas com sabores diferentes. Todas as pizzas nessa pizzaria são servidas em **12 fatias** iguais. Amanda comeu $\frac{1}{6}$ de uma pizza, Bruno comeu $\frac{3}{4}$ de outra, e Caio comeu $\frac{2}{3}$ da pizza que pediu.

\begin{tabular}{m{.3\textwidth}m{.3\textwidth}m{.3\textwidth}}
 
\begin{tikzpicture}
\fill[light, opacity = .8] (0,0) -- (30:20) arc (30:90:20) --cycle;
\foreach \x in {0,60,120}{ \draw (\x:20) -- (\x:-20);}
\foreach \x in {30,90,150}{ \draw[very thick, light] (\x:20) -- (\x:-20);}
\draw[|-|] (30:25) arc (30:90:25);
\node[] at (60:30) {$\dfrac{1}{6}$};
\draw (0,0) circle (20);
\end{tikzpicture}

&
\begin{tikzpicture}
\fill[common, opacity = .8] (0,0) -- (-180:20) arc (-180:90:20) --cycle;
\foreach \x in {0,30,60,120,150}{ \draw (\x:20) -- (\x:-20);}
\foreach \x in {0,90}{ \draw[very thick, common] (\x:20) -- (\x:-20);}
\draw[|-|] (0:25) arc (0:90:25);
\node[] at (45:30) {$\dfrac{1}{4}$};
\draw (0,0) circle (20);
\end{tikzpicture}
&
\begin{tikzpicture}
\fill[special, opacity = .8] (0,0) -- (-150:20) arc (-150:90:20) --cycle;
\foreach \x in {0,30,60,90,120,150}{ \draw (\x:20) -- (\x:-20);}
\foreach \x in {-30,90,210}{ \draw[very thick, special] (0,0) -- (\x:20);}
\draw[|-|] (-30:25) arc (-30:90:25);
\node[] at (30:30) {$\dfrac{1}{3}$};
\draw (0,0) circle (20);
\end{tikzpicture}
\\
 Fração de pizza consumida por Amanda $\frac{1}{6}$  & Fração de pizza consumida por Bruno $\frac{3}{4}$  & Fração de pizza consumida por Caio $\frac{2}{3}$ 
\end{tabular}

\begin{enumerate}[a)]
\item  Que fração de uma pizza cada fatia representa?
 \item Complete os espaços (numeradores) a seguir registrando outra representação para a fração de uma pizza que cada uma das crianças comeu.\\ Amanda: $\frac{1}{6} =\frac{}{12}  \quad \quad$ Bruno: $\frac{3}{4} =\frac{}{12} \quad \quad$ Caio: $\frac{2}{3} =\frac{}{12}$ 
 \item Quem comeu mais pizza? Quem comeu menos pizza?
 \item Que quantidade de pizza Bruno comeu a mais do que Caio?
 \item Que quantidade de pizza Amanda e Bruno comeram juntas?
  \item Que fração de uma pizza Amanda comeu a menos do que Caio?
  \item Quanto a mais de pizza Bruno consumiu, em relação a Amanda?
\end{enumerate}


\section{ORGANIZANDO AS IDEIAS }

No caso de quantidades expressas por meio de frações de uma unidade dada, para comparar, determinar a soma ou determinar a diferença, é necessário uma {\bf subdivisão da unidade} com a qual seja possível expressar ambas as quantidades. Por exemplo:
\begin{itemize} %s
  \item     Na Atividade 3, a subdivisão da unidade considerada, barra de chocolate permitiu expressar as quantidades de chocolate comidas por Alice e por Miguel a partir da contagem da mesma subdivisão da unidade. A partir dessa estratégia foram determinadas a quantidade de chocolate comidas por Alice e Miguel juntos, bem como a quantidade de chocolate restante. 
  \item     Na Atividade 4, a unidade é uma pizza e a fatia de pizza é uma subdivisão dessa unidade. Neste caso, pôde-se expressar todas as frações de pizza consumidas por Amanda, Bruno e Caio a partir contagem dessas fatias (subdivisões da unidade). Relembrando:
\end{itemize} %s

$$\dfrac{1}{6} = \dfrac{2}{12} 	\quad \quad \dfrac{3}{4} = \dfrac{9}{12} \quad \quad \dfrac{2}{3} = \dfrac{8}{12}.$$


Como os exemplos acima ilustram, a escolha adequada de uma subdivisão da unidade que permita representar as frações dadas com um mesmo denomindador foi a estratégia usada para calcular a adição e a subtração dessas frações. É exatamente essa estratégia que usaremos para calcular adição e subtração de frações em geral. 

$$\dfrac{1}{6} + \dfrac{3}{4} = \dfrac{2}{12} + \dfrac{9}{12} = \dfrac{11}{12}.$$


\section{MÃO NA MASSA }

\subsection{Atividade}

Tendo como unidade um mesmo retângulo, as representações das frações $\frac{3}{5}$ e $\frac{7}{10}$ estão ilustradas nas figuras a seguir. 

\begin{center}
\begin{tikzpicture}[scale=4]
\fill[fill=common, fill opacity=.3] (0,0) rectangle (10,5);
\fill[special] (0,0) rectangle (6,5);
\draw (0,0) rectangle (10,5);
\foreach \x in {2,4,...,8} \draw (\x,0) -- (\x, 5);

\begin{scope}[shift={(14,0)}]
\fill[fill=common, fill opacity=.3] (8,2.5) rectangle (10,5);
\fill[fill=common, fill opacity=.3] (6,0) rectangle (10,2.5);
\fill[light, opacity = .8] (0,0) rectangle (6,5);
\fill[light, opacity = .8] (6,2.5) rectangle (8,5);
\draw (0,0) rectangle (10,5);
\foreach \x in {2,4,...,8} \draw (\x,0) -- (\x, 5);
\draw (0,2.5) -- (10, 2.5);
\end{scope}
\end{tikzpicture}
\end{center}

\begin{enumerate} [\quad a)] %s
  \item     Determine uma subdivisão da unidade que permita expressar essas quantidades por frações com um mesmo denominador. Represente, usando essas figuras, essa subdivisão.
  \item     Escreva frações iguais a     $\frac{3}{5}$     e a     $\frac{7}{10}$     a partir dessa subdivisão.
  \item     Existe alguma outra subdivisão, diferente da que você usou para responder os itens a) e b), com a qual também seja possível responder ao item b)? Se sim, qual? 
  \item     Juntas, as regiões destacadas em vermelho e em bege determinam um região total maior, menor ou igual a um retângulo? Explique
\end{enumerate} %s

\subsection{Atividade}

Aqui retomamos a Atividade 2, na qual a professora Estela comprou e dividiu fitas de mesmo tamanho: a vermelha em três pedaços; a azul em dois pedaços e a amarela em quatro pedaços. 


\begin{center}
\begin{tikzpicture}[x=1.0cm,y=1.0cm, scale=.5]
\draw[fill=attention] (0.,1) rectangle (12.,3.); 
\foreach \x in {4,8} \draw[dashed] (\x,1) -- (\x,3);
\draw[fill=common] (0.,-2) rectangle (12.,0.);
\draw[dashed] (6,-2) -- (6,0);
\draw[fill=yellow,fill opacity=.3] (0.,-5) rectangle (12.,-3.);
\foreach \x in {3,6,9} \draw[dashed] (\x,-5) -- (\x,-3);
\end{tikzpicture}
\end{center}


\begin{enumerate}[a)]
  \item  Agora, a professora Estela juntou um pedaço da fita vermelha com um pedaço da fita azul. Essa nova fita formada tem tamanho maior ou menor ou igual ao tamanho orignal das fitas? A que fração de uma fita original corresponde a nova fita vermelha e azul? Qual é a diferença entre os tamanhos de uma fita original e da fita vermelha e azul?
  \item  A professora formou então mais uma fita colorida, agora juntando (de forma intercalada) dois pedaços vermelhos e três pedaços amarelos. Essa nova fita vermelha e amarela é maior ou menor do que uma fita original? A que fração de uma fita original corresponde a nova fita vermelha e azul? Qual é a diferença entre os tamanhos da fita original e da fita vermelha e amarela?
\end{enumerate}


\subsection{Atividade}


Em cada um dos itens a seguir, escreva frações iguais às frações dadas que tenham mesmo denominador. Para cada par de frações, destaque a subdivisão escolhida da unidade para determinar o denominador comum e represente essa subdivisão por meio de um desenho.

\begin{center}
  \begin{tabular}{m{0.25\textwidth}m{0.25\textwidth}m{0.25\textwidth}}
    
     a) $\frac{1}{3}$ e $\frac{2}{9}$  &   b) $\frac{3}{10}$ e $\frac{4}{5}$  &   c) 1 e $\frac{3}{7}$  \\
     \\
     d) $\frac{3}{5}$ e $\frac{8}{3}$  &   e) $\frac{7}{8}$ e $\frac{13}{12}$  &  f) $\frac{7}{4}$ e 5     
  \end{tabular}
\end{center}

\subsection{Atividade}

Faça as contas a seguir. Represente cada umas das subdivisões usadas para fazer as contas por meio de um desenho.

\begin{center}
  \begin{tabular}{m{0.25\textwidth}m{0.25\textwidth}m{0.25\textwidth}}    
     a) $\frac{1}{3} - \frac{2}{9}$  &   b) $\frac{3}{10} + \frac{4}{5}$  &   c) $1 - \frac{3}{7}$     
  \end{tabular}
\end{center}

\subsection{Atividade}
Miguel deseja calcular a soma $2 + \frac{1}{3}$. Para isso, marcou na reta numérica ponto determinado pela justaposição do segmento correspondente a $2$ unidades com um segmento igual a $\frac{1}{3}$ da unidade, como na figura abaixo. 

Miguel relacionou essa estratégia com o seguinte cálculo: 
$$ 2 + \frac{1}{3} =  \frac{6}{3} + \frac{1}{3} = \frac{7}{3}$$

% 
% \begin{center}
%  \begin{tikzpicture}[x=17mm,y=17mm]
%   \draw[->] (0,-.25) -- (0,3.25);
%   \foreach \x in {0,...,3}{
%   \draw (-3pt,\x)--(3pt,\x);
%   \node at (-7pt,\x) {\x};}
%  \foreach \x in {2+1/3,2+2/3}\draw (-2pt,\x)--(2pt,\x); 
%  \draw[|-|] (9pt,2) -- (9pt,2+1/3);
%  \node at (20pt,2+1/6) {$\dfrac{1}{3}$};
%  \draw[->] (-35pt,2+1/3) -- (-9pt,2+1/3);
%  \node at (-1.1,2+1/3) {$2 + \dfrac{1}{3}$};
%  \fill[common] (0,2+1/3) circle (3pt);
%  \end{tikzpicture}
% \end{center}


\begin{center}
 \begin{tikzpicture}[x=17mm,y=17mm]
  \draw[->] (0,-.25) -- (0,3.25);
  \foreach \x in {0,...,3}{
  \draw (-3pt,\x)--(3pt,\x);
  \node at (-7pt,\x) {\x};}
 \foreach \x in {2+1/3,2+2/3}\draw (-2pt,\x)--(2pt,\x); 
 \draw[|-|] (9pt,2) -- (9pt,2+1/3);
 \node at (20pt,2+1/6) {$\dfrac{1}{3}$};
 \draw[->] (-35pt,2+1/3) -- (-9pt,2+1/3);
 \node at (-1.1,2+1/3) {$2 + \dfrac{1}{3}$};
 \fill[common] (0,2+1/3) circle (3pt);
 \draw[dotted] (9 pt, 2+1/3) -- (1.9, 2+1/3);

 \begin{scope}[shift={(2,0)}]
 %reta numerica vertical
 \draw[->] (0,-.25) -- (0,3.25);
  \foreach \x in {0,...,3}{
  \draw (-3pt,\x)--(3pt,\x);
  \node at (-7pt,\x) {\x};}
\foreach \x in {0,.3333,...,2.6666}\draw (-2pt,\x)--(2pt,\x); 
  \fill[common] (0,2+1/3) circle (3pt);

% segmentos de 1/3 ao lado da reta

\foreach \x in {1,...,6}{ 
\draw[|-|] (9pt,\x/3+.01) -- (9pt,\x/3+1/3-.01);
\node at (20pt,\x/3+1/6) {{\small $\frac{1}{3}$}};}
\draw[|-|] (9pt,0) -- (9pt,1/3-.01);
\node at (20pt,1/6) {{\small $\frac{1}{3}$}};

%flecha e texto.
\draw[<-]  (30 pt, 2+1/6) -- (56pt, 2+1/6);
\node at (90pt, 2+1/6) {1 fração de $\dfrac{1}{3}$};
\node at (90pt, 1.6) {$+$};
\node at (90pt, 1) {6 frações de $\dfrac{1}{3}$};
%linhas tracejadas e chave
\foreach \x in {0,2} \draw[dotted] (25pt,\x) -- (45pt,\x);
\draw [thick, decoration={brace,mirror,raise=5}, decorate] (45pt,0) -- (45pt,2); 
 \end{scope}
 \end{tikzpicture}
\end{center}


\begin{enumerate}[a)]
 \item Em cada item a seguir, a partir da imagem repita o procedimento feito por Miguel e realize os cálculos.
 
\begin{center} 
\begin{tabular}{m{.3\textwidth}m{.3\textwidth}m{.3\textwidth}}
 (A) & (B) & (C)\\
  
 \begin{tikzpicture}[x=17mm,y=17mm]
  \draw[->] (0,-.5) -- (0,4.5);
  \foreach \x in {0,...,4}{
  \draw (-3pt,\x)--(3pt,\x);
  \node at (-7pt,\x) {\x};}
 \foreach \x in {3.25,3.5,3.75}\draw (-2pt,\x)--(2pt,\x); 
 \fill[common] (0,3.25) circle (3pt);
 
 % setinha e texto
 \draw[->] (-35pt,3.25) -- (-9pt,3.25);
 \node at (-1.1,3.25) {$3 + \dfrac{1}{4}$};
 
 \end{tikzpicture}
& 

 \begin{tikzpicture}[x=17mm,y=17mm]
  \draw[->] (0,-.5) -- (0,5.5);
  \foreach \x in {0,...,5}{
  \draw (-3pt,\x)--(3pt,\x);
  \node at (-7pt,\x) {\x};}
 \draw (-2pt,4.5)--(2pt,4.5); 
 \fill[common] (0,4.5) circle (3pt);
 
 % setinha e texto
 \draw[->] (-35pt,4.5) -- (-9pt,4.5);
 \node at (-1.1,4.5) {$4 + \dfrac{1}{2}$};
  \end{tikzpicture}
 &
 \begin{tikzpicture}[x=17mm,y=17mm]
  \draw[->] (0,-.5) -- (0,3.5);
  \foreach \x in {0,...,3}{
  \draw (-3pt,\x)--(3pt,\x);
  \node at (-7pt,\x) {\x};}
 \draw (-2pt,2.6)--(2pt,2.6); 
 \foreach \x in {2.2,2.4,...,2.8}\draw (-2pt,\x)--(2pt,\x); 
 \fill[common] (0,2.6) circle (3pt);
 
 % setinha e texto
 \draw[->] (-35pt,2.6) -- (-9pt,2.6);
 \node at (-1.1,2.6) {$2 + \dfrac{3}{5}$};
 
 \end{tikzpicture}
\end{tabular}
\end{center}
 
 \item Que valor é obtido se juntarmos 7 inteiros com dois terços?
\end{enumerate}

\subsection{Atividade}


Quanto se deve acrescentar a $\frac{3}{8}$ para que se obtenha $\frac{27}{8}$?


\subsection{Atividade}

Qual o maior número, $\frac{19}{7}$ ou $2$? Quanto se deve acrescentar ao menor número para chegar ao maior?

\subsection{Atividade}

Observando a reta, Miguel conseguiu determinar o tamanho do segmento entre os dois pontos $A = 3$ e $B = 7$ marcados da seguinte forma:

\begin{center}
\definecolor{DarkGreen}{rgb}{0.0, 0.5, 0.0}
\begin{tikzpicture}[x=17mm,y=17mm]
\draw[->] (-0.5,0) -- (7.5,0) ; %reta anterior
\foreach \x in {0,...,7}{ \draw (\x,3pt) -- (\x,-3pt) node[below] {\x}; }
\draw[common, line width=0.4mm] (3,0) -- (7,0);
\foreach \x in {3,7} \fill[common] (\x,0) circle (3 pt); 
\node[above] at (3,3pt) {$A$};  
\node[above] at (7,3pt) {$B$};
\node[color=attention] at (5,-30pt) {{\Large 7}};
\node[] at (5.2,-30pt) {{\Large $-$}};
\node[color=DarkGreen] at (5.4,-30pt) {{\Large 3}};
\node[] at (5.6,-30pt) {{\Large $=$}};
\node[common] at (5.8,-30pt) {{\Large 4}};
\draw[|-|, shift={(0,-30pt)},  color=DarkGreen, line width=0.4mm] (0,0)--(3,0);
\draw[|-|, shift={(0,-45pt)}, color=attention, line width=0.4mm] (0,0)--(7,0);
\end{tikzpicture}
\end{center}

Miguel calculou o tamanho do segmento azul fazendo a diferença entre o tamanho do segmento vermelho e o tamanho do segmento verde. Assim, concluiu que o tamanho do segmento AB é igual a 4.
Usando um raciocínio parecido, e considerando $C = \frac{5}{4}$ e $D=\frac{11}{6}$, ajude Miguel a realizar as tarefas a seguir.

\begin{center}
\begin{tikzpicture}[x=50mm,y=50mm]
\draw[->] (-0.25,0) -- (2.25,0) ; %reta anterior
\foreach \x in {0,1,2}{ \draw (\x,3pt) -- (\x,-3pt) node[below] {\x}; }
\draw[common, line width=0.4mm] (5/4,0) -- (11/6,0);
\foreach \x in {5/4,11/6} \fill[common] (\x,0) circle (3 pt); 
\node[above] at (5/4,3pt) {$C$};  
\node[above] at (11/6,3pt) {$D$};
\node[below] at (5/4,-3pt) {$\frac{5}{4}$};  
\node[below] at (11/6,-3pt) {$\frac{11}{6}$};
\end{tikzpicture}
\end{center}


\begin{enumerate} [\quad a)] %s
  \item     Escreva C e D a partir de uma mesma subdivisão da unidade (isto é, com o mesmo denominador).
  \item     Determine seis frações que correspondam a pontos entre C e D.     \mbox{} \newline      Discuta com seus colegas se é possível determinar mais que seis valores e, se for possível, qual seria a estratégia para fazer isso.
  \item     Calcule o tamanho do segmento     $CD$    .
  \item     Determine uma fração que somada a     $\frac{5}{4}$     dê um resultado menor que     $\frac{11}{6}$    . Justifique a sua resposta usando a reta.     $$ \dfrac{5}{4} +\dfrac{\quad}{} = \dfrac{11}{6}.$$     
  \item     Encontre mais três frações possíveis para completar a expressão do item anterior. 
  \item     Determine duas frações possíveis que quando somadas a     $\frac{5}{4}$     tenham como resultado     $\frac{11}{6}$    . Justifique a sua resposta usando a reta.     $$ \dfrac{5}{4} +\dfrac{\quad}{} + \dfrac{\quad}{} = \dfrac{11}{6}.$$     
\end{enumerate} %s

\subsection{Atividade}

A família de Miguel reservou um determinado espaço retangular para fazer um canteiro em seu quintal. A família quer que o cateiro tenha rosas e verduras frescas. O pai de Miguel disse que precisa de $\frac{2}{3}$ do espaço inicialmente reservado, para cultivar rosas. A mãe disse que necessita de $\frac{1}{2}$ desse espaço, para plantar as verduras. Quando Miguel ouviu o diálogo dos pais, pensou nas seguintes questões:
\begin{enumerate} [\quad a)] %s
  \item     Quem precisa de mais espaço, seu pai ou sua mãe? 
  \item     O espaço reservado inicialmente para o canteiro é suficiente para comportar os espaços de que o pai e a mãe de Miguel precisam? 
  \item     Caso o espaço seja suficiente, que fração do mesmo ficaria sem uso? 
  \item     Caso o espaço não seja suficiente, que fração do canteiro reservado inicialmente deverá ser acrescentada para que a família consiga fazer as plantações que deseja? 
\end{enumerate} %s


Faça um desenho que ajude a explicar as suas respostas para as questões de Miguel. Não deixe de indicar a subdivisão da unidade que você empregou.

\subsection{Atividade}

Há três recipientes cilíndricos, de mesmo tamanho, contendo água. No primeiro recipiente, a água ocupa dois terços de sua capacidade. No segundo, a água ocupa metade de sua capacidade. No terceiro, a água ocupa cinco oitavos de sua capacidade.  
\begin{imagem*}[breakable]{}{}  
  figura artística  
  
    \includegraphics[width=240pt, keepaspectratio]{../../livro/media//cap5/ativ/garrafas.png}  
\end{imagem*}

É possível redistribuir a água de todos os recipientes em somente dois deles?

\section{QUEBRANDO A CUCA }

\subsection{Atividade}

Diga se as afirmações a seguir são verdadeiras ou falsas. Para as verdadeiras, explique com as suas palavras por que acha que são verdadeiras. Para as falsas, dê um exemplo que justifique a sua avaliação.
\begin{enumerate} [\quad a)] %s
  \item     A soma de um número inteiro com uma fração não inteira é, necessariamente, um número inteiro.
  \item     A diferença entre um número inteiro e uma fração não inteira é, necessariamente, um número inteiro.
  \item     A soma de uma fração não inteira com um número inteiro é, necessariamente, uma fração não inteira.
  \item     A diferença entre uma fração não inteira e uma fração não inteira é, necessariamente, uma fração não inteira.
  \item     A diferença entre uma fração não inteira e uma fração não inteira pode ser uma fração não inteira.
\end{enumerate} %s

