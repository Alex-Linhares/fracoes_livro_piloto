\documentclass[a4,12pt]{book}

\usepackage{template}

\begin{document}

\chapter{Multiplicando a fração da unidade }


\begin{imagem*}[breakable]{}{}   FIGURA ARTÍSTICA  
  HISTÓRIA EM QUADRINHOS  
  
  Miguel é um menino negro de cabelos cheios e cacheados.  
  
  Alice é uma menina morena de rabo de cavalo. Ela já apareceu numa atividade da Lição 1 é a mesma personagem.  
  
  {\bf Quadrinho 1}  
  
  Alice: Oi Miguel! Por que você faltou a aula passada? A professora falou de frações.  
  
  Miguel: Eu tive febre.   
  
  {\bf Quadrinho 2}  
  
  Miguel escrevendo no quadro as frações abaixo para mostrar para Alice  
  
  Miguel: Mas a minha mãe me ensinou frações em casa. Tem o   $\frac{1}{2}$  ,   $\frac{1}{3}$  ,   $\frac{1}{4}$   até   $\frac{1}{10}$  .  
  
  Alice: Não foi isso o que vimos aqui. A gente repartiu figuras de papel e outros objetos. Tinha que ser em partes iguais ou com a mesma quantidade em cada parte! Aí surgiram nomes: se forem duas partes iguais, cada uma delas é metade da coisa, isso a gente já sabia. Se forem três partes iguais, cada uma é um terço ou a terça parte do que foi repartido e assim vai.  
  
  {\bf Quadrinho 3}  
  
  Quadro negro escrito por Miguel:  
  
  Frações  
  
  meio   $\longrightarrow \frac{1}{2}$     
  
  terço   $\longrightarrow \frac{1}{3}$  
  
  quarto   $\longrightarrow \frac{1}{4}$  
  
  $\cdots$  
  
  décimo   $\longrightarrow \frac{1}{10}$  
  
  Alice com expressão zangada.  
  
  Miguel olhando para Alice  
  
  Miguel: Isso mesmo! Minha mãe falou que um terço é   $\frac{1}{3}$  , que um quarto é   $\frac{1}{4}$  , um quinto é   $\frac{1}{5}$  .  
  
  Alice: Esse negócio não parece estar certo. Os números ficam um ao lado do outro, 10, 11, 12,   $\cdots$   e não um embaixo do outro como você mostrou aí!  
  
  {\bf Quadrinho 4}  
  A professora aparece no quadro  
  
  Miguel: Minha mãe sabe do que está falando. Hoje no ônibus ela me mostrou frações no painel do motorista, ontem em casa ela mostrou uma seringa e um copo com marquinhas e lá estavam as frações também.  
  
    \includegraphics[width=120pt, keepaspectratio]{../../livro/media//cap2/secoes/boia-tubular-compativel-com-marcador-de-combustivel-nauticos-14480-mlb2755833842_052012-o.jpg}  
  
    \includegraphics[width=60pt, keepaspectratio]{../../livro/media//cap2/secoes/3ccsyringeminim-02.jpg}  
  
    \includegraphics[width=120pt, keepaspectratio]{../../livro/media//cap2/secoes/jarra-de-vidro-graduada-300x199.jpg}  
  
  
  Alice: Mas a professora não falou disso...  
  
  Professora: Crianças, não briguem, os dois estão certos. Vamos falar disso na lição de hoje.  
  
\end{imagem*}




\section*{EXPLORANDO O ASSUNTO }



\subsection{Atividade}

O pai de Ana, Beatriz e Clara trouxe duas barras de chocolate para serem repartidas entre elas.
\begin{center}
\includegraphics[width=240pt, keepaspectratio]{../../livro/media//cap2/secoes/explorando/mais/de/uma/chocolate-01.jpg} 
\end{center}

Ana propôs que cada barra fosse dividida em três partes iguais e que cada irmã ficasse com duas dessas partes.
\begin{center}
\includegraphics[width=354pt, keepaspectratio]{../../livro/media//cap2/secoes/explorando/mais/de/uma/chocolate-03.png} 
\end{center}

\begin{enumerate}[a)] %s
  \item Na divisão de cada uma das barras de chocolate em três partes iguais, cada parte é que fração de uma barra de chocolate?
  \item Você concorda com a divisão que Ana sugeriu? Explique. 
  \item Com essa divisão, as três irmãs receberiam a mesma quantidade de chocolate? 
  \item Na divisão proposta por Ana, como você nomearia, usando uma fração de uma barra de chocolate, a quantidade de chocolate que cada irmã receberia? 
\end{enumerate}
  
Ana não quer o chocolate e decidiu dar a quantidade de chocolate que recebeu na divisão das barras para as suas irmãs.

\begin{enumerate}[e)]
\item Se Ana desse metade da quantidade de chocolate que recebeu para cada uma de suas irmãs, que quantidade de chocolate Beatriz e Clara passariam a ter? Como você nomearia, usando frações, essas quantidades?  
\item[f)] E se Ana desse toda a quantidade de chocolate que recebeu para Beatriz, que quantidade de chocolate  Beatriz passaria a ter? Como você nomearia, usando frações, essa quantidade?
\end{enumerate} %s


\subsection{Atividade}

Um grupo de cinco amigos (Amarildo, Beto, Carlos, Davi e Edilson) encomendou três tortas salgadas para uma comemoração.

\begin{center}
 \begin{tikzpicture}[x=1mm,y=1mm]
  \draw (0,0) rectangle (60,30);
  \draw (70,0) rectangle (130,30);
 \end{tikzpicture}
  
\end{center}
 
\begin{enumerate} [\quad a)] %s
  \item     Como dividir as três tortas de modo que cada amigo receba a mesma quantidade de torta? Faça um desenho no seu caderno mostrando sua proposta de divisão. Indique qual parte é de qual amigo!
  \item     Considerando-se uma torta, como você nomearia, usando frações, a quantidade de torta que:     
\begin{enumerate} [\quad I)] %d
      \item         Amarildo recebeu? 
      \item         Amarildo e Beto receberam juntos? 
      \item         Amarildo, Beto e Carlos receberam juntos? 
      \item         Amarildo, Beto, Carlos e Davi receberam juntos?
      \item         Amarildo, Beto, Carlos, Davi e Edilson receberam juntos?
\end{enumerate} %d

  \item     A quantidade de torta que cada amigo recebeu é menor do que um quinto de torta? E do que dois quintos de torta? Explique sua resposta.
  \item     A quantidade de torta que cada amigo recebeu é maior do que três quintos de torta? E do que quatro quintos de torta? Explique sua resposta.
\end{enumerate} %s


\subsection{Atividade}

Para a sobremesa do almoço de domingo, papai passou em uma confeitaria em que as tortas estavam divididas em 8 fatias, como na figura abaixo. 

\begin{imagem*}[breakable]{}{}   FIGURA ARTÍSTICA - Imagem de três tortas circulares idênticas cortadas em 8 fatias iguais cada uma dentro do balcão de vidro de uma confeitaria. Atenção: também há figura na resposta.  
\end{imagem*}

\begin{enumerate} [\quad a)] %s
  \item     Que fração de uma torta é uma fatia? Explique.
  \item     Domingo papai comprou 4 fatias, quantos oitavos de uma torta havia para a sobremesa?
  \item     Na pergunta anterior, apresente outra fração que represente a quantidade de torta que papai comprou. Explique sua resposta.
  \item     Hoje papai comprou 10 fatias de torta. Como podemos representar essa quantidade de torta em termos de frações de     {\bf uma torta}    ? Lembre-se que oito fatias formam uma torta inteira.
\end{enumerate} %s

\subsection{Atividade}

Complete as afirmações com uma das frações: ``dois meios'', ``dois terços'', ``dois quintos'', ``três quartos'', ``oito sextos'' e ``nove meios'' para que sejam verdadeiras.

    \includegraphics[width=420pt, keepaspectratio]{../../livro/media//cap2/secoes/preencha-01.png}  

\begin{enumerate}[a)]
 \item A parte pintada de vermelho em \begin{tikzpicture}[scale=0.8]
                           \draw[fill=attention] (0,0) rectangle (20,10);
                           \draw (20,0) rectangle (30,10);
                          \end{tikzpicture} 
                          é \begin{tikzpicture}
                             \draw (0,0) -- (20,0);
                            \end{tikzpicture}
                            de \begin{tikzpicture}[scale=0.8]
                                 \draw (0,0) rectangle (30,10);
                               \end{tikzpicture}.

 \item A parte pintada de vermelho em \begin{tikzpicture}
                           \draw[fill=attention] (0,0) rectangle (8,8);
                           \draw (0,0) -- (8,8);
                          \end{tikzpicture} 
                          é \begin{tikzpicture}
                             \draw (0,0) -- (20,0);
                            \end{tikzpicture}
                            de \begin{tikzpicture}
                            \draw (0,0) rectangle (8,8);                           
                           \end{tikzpicture}. 


 \item A parte pintada de vermelho em \begin{tikzpicture}
                           \fill[attention] (0,0) -- (72:4) arc (72:-72:4) --cycle;
                           \draw (0,0) circle (4);
                           \foreach \x in {0,72,...,288}{
                           \draw (0,0) -- (\x:4);}                           
                          \end{tikzpicture} 
                          é \begin{tikzpicture}
                             \draw (0,0) -- (20,0);
                            \end{tikzpicture}
                            de \begin{tikzpicture}
                            \draw (0,0) circle (4);                           
                           \end{tikzpicture}.            

 \item A parte pintada de vermelho em \begin{tikzpicture}
                                       \fill[attention]  \foreach \x/\y in {36/72,108/144,180/212, 252/284, 324/360}{ (0,0) -- (\x-18:4) -- (\y-18:2)--(\x-18 +72:4) -- (0, 0)};
                                       \draw  \foreach \x/\y in {36/72,108/144,180/212, 252/284, 324/360}{ (\x-18:4) -- (\y-18:2)--(\x-18 +72:4)};
                                      \end{tikzpicture} \begin{tikzpicture}
                                       \fill[attention]  \foreach \x/\y in {36/72,108/144,180/212, 252/284, 324/360}{ (0,0) -- (\x-18:4) -- (\y-18:2)--(\x-18 +72:4) -- (0, 0)};
                                       \draw  \foreach \x/\y in {36/72,108/144,180/212, 252/284, 324/360}{ (\x-18:4) -- (\y-18:2)--(\x-18 +72:4)};
                                      \end{tikzpicture} \begin{tikzpicture}
                                       \fill[attention]  \foreach \x/\y in {36/72,108/144,180/212, 252/284, 324/360}{ (0,0) -- (\x-18:4) -- (\y-18:2)--(\x-18 +72:4) -- (0, 0)};
                                       \draw  \foreach \x/\y in {36/72,108/144,180/212, 252/284, 324/360}{ (\x-18:4) -- (\y-18:2)--(\x-18 +72:4)};
                                      \end{tikzpicture} \begin{tikzpicture}
                                       \fill[attention]  \foreach \x/\y in {36/72,108/144,180/212, 252/284, 324/360}{ (0,0) -- (\x-18:4) -- (\y-18:2)--(\x-18 +72:4) -- (0, 0)};
                                       \draw  \foreach \x/\y in {36/72,108/144,180/212, 252/284, 324/360}{ (\x-18:4) -- (\y-18:2)--(\x-18 +72:4)};
                                      \end{tikzpicture} \begin{tikzpicture}
                                       \fill[attention]  \foreach \x/\y in {36/72,108/144,180/212, 252/284, 324/360}{ (0,0) -- (\x-18:4) -- (\y-18:2)--(\x-18 +72:4) -- (0, 0)};
                                       \draw  \foreach \x/\y in {36/72,108/144,180/212, 252/284, 324/360}{ (\x-18:4) -- (\y-18:2)--(\x-18 +72:4)};
                                      \end{tikzpicture} 
                           
\end{enumerate}
    






\section*{ORGANIZANDO AS IDEIAS }








Se uma torta está dividida em três partes iguais, a torta fica separada em três terços. Assim, como visto na historinha do início da lição, tanto faz escrever: ``$\dfrac{1}{3}$ da torta'' ou ``um terço da torta'' para se referir à fatia destacada na figura.

\begin{imagem*}[breakable]{}{}   FIGURA GEOMÉTRICA - incluindo o texto  \mbox{} \newline        \includegraphics[width=180pt, keepaspectratio]{../../livro/media//cap2/secoes/dois_e_tres_tercos_torta.jpg}\end{imagem*}


Duas fatias são ``dois terços da torta'', o que pode ser expresso simplesmente por ``$\dfrac{2}{3}$ da torta''. Deste modo, é claro que ``três terços da torta'' é uma torta inteira.

\begin{imagem*}[breakable]{}{}   FIGURA GEOMÉTRICA - incluindo o texto  \mbox{} \newline        \includegraphics[width=300pt, keepaspectratio]{../../livro/media//cap2/secoes/um_terco_torta.jpg}  
  Incluir abaixo da imagem de   $\frac{3}{3}$   da torta:   $\frac{3}{3}$   da torta = 1 torta.\end{imagem*}

Também pode-se considerar quatro terços, cinco terços ou seis terços da torta, basta juntar novos terços à torta inteira.

\begin{imagem*}[breakable]{}{}    FIGURA GEOMÉTRICA - incluindo o texto  \mbox{} \newline         \includegraphics[width=420pt, keepaspectratio]{../../livro/media//cap2/secoes/20160419_230500.jpg}  
  Incluir abaixo dessas imagens, respectivamente:  
  $\frac{4}{3}$   da torta = 1 torta e   $\frac{1}{3}$   da torta;  
  $\frac{5}{3}$   da torta = 1 torta e   $\frac{2}{3}$   da torta;  
  $\frac{6}{3}$   da torta = 2 tortas.  
\end{imagem*}



Se uma torta é repartida em três partes iguais, cada fatia é um terço da torta - ou, simplesmente, $\frac{1}{3}$ da torta. Juntando essas fatias, é possível se ter dois terços ($\frac{2}{3}$) e três terços ($\frac{3}{3}$) da torta. Com mais do que uma torta repartida em três partes iguais, obtem-se quatro terços ($\frac{4}{3}$), cinco terços ($\frac{5}{3}$), seis terços ($\frac{6}{3}$), etc, da torta. Na representação simbólica, as frações que registram essas quantidades têm o número 3 ``abaixo'' do traço de fração, e, por isso, são denominadas terços. O número que informa a parte da unidade que ``dá nome'' à fração é chamado de {\it denominador} da fração. Assim, nas frações $\frac{1}{3}$, $\frac{2}{3}$, $\frac{3}{3}$,  $\frac{4}{3}$ e $\frac{5}{3}$, o 3 é o denominador, identificando ``terços''. 


Já o número que aparece ``acima'' do traço de fração informa quantos terços estão sendo considerados. Esse número é chamado de {\it numerador} da fração. Por exemplo, na fração $\frac{1}{3}$ o numerador é 1 e na fração $\frac{4}{3}$ o numerador é 4.

Essa mesma forma de nomear vale para outras frações, mesmo que o denominador seja diferente de 3:\mbox{} \newline 
Em $\frac{2}{5}$, por exemplo, o numerador é 2 e o denominador é 5. Fala-se {\it dois quintos}.\mbox{} \newline 
Em $\frac{10}{8}$, por exemplo, o numerador é 10 e o denominador é 8. Fala-se {\it dez oitavos}. \mbox{} \newline 
Como você pôde observar, a nomeação de uma fração depende fortemente do denominador da fração. Para ler a fração basta lermos o {\bf número} do numerador seguido do {\bf nome que identifica a fração do tipo} $\frac{1}{b}$, nessa ordem. Veja:

$$\frac{1}{3}\rightarrow \text{ um terço;} \quad \frac{2}{3}\rightarrow \text{ dois terços;} \quad \frac{5}{3}\rightarrow \text{ cinco terços;}$$
$$\frac{1}{8}\rightarrow \text{ um oitavo;} \quad \frac{3}{8}\rightarrow \text{ três oitavos;} \quad \frac{7}{8}\rightarrow \text{ sete oitavos.}$$
Anote agora os nomes de algumas outras frações:
$$\frac{1}{2}\rightarrow \text{  um meio;} \quad \frac{1}{3}\rightarrow\text{  um terço;} \quad \frac{1}{4}\rightarrow\text{  um quarto;}$$
$$\frac{1}{5}\rightarrow\text{  um quinto;}\quad \frac{1}{6}\rightarrow\text{  um sexto;} \quad \frac{1}{7}\rightarrow\text{  um sétimo;}$$
$$\frac{1}{8}\rightarrow\text{  um oitavo;}\quad \frac{1}{9}\rightarrow\text{  um nono;}\quad \frac{1}{10}\rightarrow\text{  um décimo.}$$

Para a fração $\frac{1}{11}$, fala-se um onze avos. Da mesma forma, são nomeadas frações cujo denominador é maior do que 11. Por exemplo:
$$\frac{1}{12}\rightarrow \text{  um doze avos;}\quad \frac{1}{13}\rightarrow \text{ um treze avos;} \quad \frac{5}{13}\rightarrow \text{ cinco treze avos.}$$

Curioso para saber sobre o significado da palavra {\bf avos}? Pergunte ao seu professor. O importante é lembrar que, para denominadores maiores 11, acrescenta-se a expressão ``avos'' ao final da leitura da fração.

Contudo, para frações cujo denominador é uma potência de 10, usa-se outra formar de ler:
( ERRO:\{\$\} )( ERRO:\{\$\} )\textbackslash{}frac\{1\}\{100\}\textbackslash{}rightarrow \textbackslash{}text\{ um centésimo;\}\textbackslash{}quad \textbackslash{}frac\{13\}\{100\}\textbackslash{}rightarrow \textbackslash{}text\{ treze centésimos;\}\textbackslash{}quad
\textbackslash{}frac\{33\}\{1000\}\textbackslash{}rightarrow \textbackslash{}text\{ trinta e três milésimos.\}( ERRO:\{\$\} )( ERRO:\{\$\} )

{\bf Pronto! Agora você já é capaz de ler diversos tipos de frações.}





\begin{imagem*}[breakable]{}{}   FIGURA ARTÍSTICA - fazer imagem da fração como se estivesse escrita a mão por uma criança de 9 anos   \mbox{} \newline        \includegraphics[width=420pt, keepaspectratio]{../../livro/media//cap2/secoes/numerador_denominador.jpg}   \end{imagem*}


\section*{MÃO NA MASSA }



\subsection{Atividade}








Uma pizza gigante foi dividida em doze fatias iguais. 
Pedro comeu quatro fatias, Isabella cinco fatias, Bernardo duas fatias e Manuela apenas uma fatia.
\begin{imagem*}[breakable]{}{}   FIGURA GEOMÉTRICA  Também há figuras em Resposta  
\end{imagem*}

\begin{center}
  \begin{tabulary}{0.8\textwidth}{*{50}{L}}
    \hline \hline \\
                                                                                &                                  Pedro &   Isabella                             &   Bernardo                              &   Manuela                              \\
    \hline \\
     Pinte a fração de pizza consumida  por cada pessoa      &   \includegraphics[width=42pt, keepaspectratio]{../../livro/media/cap2/secoes/pizza-notacao.png} &  \includegraphics[width=42pt, keepaspectratio]{../../livro/media/cap2/secoes/pizza-notacao.png}  &   \includegraphics[width=42pt, keepaspectratio]{../../livro/media/cap2/secoes/pizza-notacao.png}  &   \includegraphics[width=42pt, keepaspectratio]{../../livro/media/cap2/secoes/pizza-notacao.png} \\
    \hline \\
     Escreva, por extenso, a fração de pizza consumida por cada pessoa                          &                                        &                                        &                                         &                                        \\
    \hline \\
     Escreva, usando notação simbólica matemática, a fração de pizza consumida por cada pessoa &                                        &                                        &                                         &                                        \\
    \hline \\
  \end{tabulary}
\end{center}

\begin{enumerate} [\quad I)] %d
  \item     Na sua opinião, qual representação de fração     ``gasta menos lápis''     para se escrita? Usando notação simbólica matemática, escrevendo por extenso ou pintando?
  \item     Na sua opinião, qual a representação que mais rapidamente ajuda a decidir quem comeu mais e quem comeu menos pizza?
\end{enumerate} %d







\subsection{Atividade}







Para cada figura a seguir, indique a fração da figura que está pintada de vermelho. Esta fração é maior, menor ou exatamente igual a $\frac{1}{2}$ da figura?
\begin{imagem*}[breakable]{}{}   FIGURA GEOMÉTRICA - Figuras devem ser feitas individualmente sem os itens.  
    \includegraphics[width=300pt, keepaspectratio]{../../livro/media//cap2/secoes/comparacao-01.png}  
\end{imagem*}







\subsection{Atividade}







Um grupo de amigos está dividindo duas pizzas circulares do mesmo tamanho. A primeira pizza foi cortada em 4 fatias de mesmo tamanho. A segunda pizza foi dividida em 8 fatias iguais.

\begin{enumerate} [\quad a)] %s
  \item     Uma fatia da primeira pizza é que fração dessa pizza? Responda usando notação simbólica matemática.
  \item     Uma fatia da segunda pizza é que fração dessa pizza? Responda usando notação simbólica matemática.
  \item     Qual fatia tem mais quantidade de pizza: uma fatia da primeira pizza ou uma fatia da segunda? Explique usando um desenho.
\end{enumerate} %s









\subsection{Atividade}







Preencha cada lacuna a seguir com uma fração adequada (use notação simbólica matemática). Perceba que uma mesma parte pintada pode ser descrita por frações diferentes com unidades diferentes.
\begin{imagem*}[breakable]{}{}   FIGURA GEOMÉTRICA - Figuras devem ser feitas individualmente sem os itens ou textos.  
\end{imagem*}
\begin{enumerate} [\quad a)] %s
  \item     A parte pintada em vermelho em         \includegraphics[height=30pt, keepaspectratio]{../../livro/media/undefined/parte-02-03-preenchido-04.png}     é \_ \_ \_   de         \includegraphics[height=30pt, keepaspectratio]{../../livro/media/cap2/secoes/parte-02-01-vazio.png}    .
  \item     A parte pintada em vermelho em         \includegraphics[height=30pt, keepaspectratio]{../../livro/media/undefined/parte-02-03-preenchido-04.png}     é \_ \_ \_   de         \includegraphics[height=30pt, keepaspectratio]{../../livro/media/cap2/secoes/parte-02-02-vazio.png}    .
  \item     A parte pintada em vermelho em         \includegraphics[height=30pt, keepaspectratio]{../../livro/media/undefined/parte-02-03-preenchido-04.png}     é \_ \_ \_   de         \includegraphics[height=30pt, keepaspectratio]{../../livro/media/cap2/secoes/parte-02-03-vazio.png}    .
  \item     A parte pintada em vermelho em         \includegraphics[height=30pt, keepaspectratio]{../../livro/media/undefined/parte-02-03-preenchido-03.png}     é \_ \_ \_   de         \includegraphics[height=30pt, keepaspectratio]{../../livro/media/cap2/secoes/parte-02-01-vazio.png}    .
  \item     A parte pintada em vermelho em         \includegraphics[height=30pt, keepaspectratio]{../../livro/media/undefined/parte-02-03-preenchido-03.png}     é \_ \_ \_   de         \includegraphics[height=30pt, keepaspectratio]{../../livro/media/cap2/secoes/parte-02-02-vazio.png}    .
  \item     A parte pintada em vermelho em         \includegraphics[height=30pt, keepaspectratio]{../../livro/media/undefined/parte-02-03-preenchido-03.png}     é \_ \_ \_   de         \includegraphics[height=30pt, keepaspectratio]{../../livro/media/cap2/secoes/parte-02-03-vazio.png}    .
  \item     A parte pintada em vermelho em         \includegraphics[height=30pt, keepaspectratio]{../../livro/media/undefined/parte-02-03-preenchido-02.png}     é \_ \_ \_   de         \includegraphics[height=30pt, keepaspectratio]{../../livro/media/cap2/secoes/parte-02-01-vazio.png}    .
  \item     A parte pintada em vermelho em         \includegraphics[height=30pt, keepaspectratio]{../../livro/media/undefined/parte-02-03-preenchido-02.png}     é \_ \_ \_   de         \includegraphics[height=30pt, keepaspectratio]{../../livro/media/cap2/secoes/parte-02-02-vazio.png}    .
  \item     A parte pintada em vermelho em         \includegraphics[height=30pt, keepaspectratio]{../../livro/media/undefined/parte-02-03-preenchido-02.png}     é \_ \_ \_   de         \includegraphics[height=30pt, keepaspectratio]{../../livro/media/cap2/secoes/parte-02-03-vazio.png}    .
  \item     A parte pintada em vermelho em         \includegraphics[height=30pt, keepaspectratio]{../../livro/media/undefined/parte-02-03-preenchido-01.png}     é \_ \_ \_   de         \includegraphics[height=30pt, keepaspectratio]{../../livro/media/cap2/secoes/parte-02-01-vazio.png}    .
  \item     A parte pintada em vermelho em         \includegraphics[height=30pt, keepaspectratio]{../../livro/media/undefined/parte-02-03-preenchido-01.png}     é \_ \_ \_   de         \includegraphics[height=30pt, keepaspectratio]{../../livro/media/cap2/secoes/parte-02-02-vazio.png}    .
  \item     A parte pintada em vermelho em         \includegraphics[height=30pt, keepaspectratio]{../../livro/media/undefined/parte-02-03-preenchido-01.png}     é \_ \_ \_   de         \includegraphics[height=30pt, keepaspectratio]{../../livro/media/cap2/secoes/parte-02-03-vazio.png}    .
\end{enumerate} %s








\subsection{Atividade}







Na tabela a seguir, pinte cada figura de modo que a parte pintada seja a fração da figura indicada na coluna à esquerda na mesma linha. Indique também, usando notação simbólica matemática, qual fração da figura ficou sem pintar.
\begin{imagem*}[breakable]{}{}   FIGURA GEOMÉTRICA - Figuras devem ser feitas individualmente sem textos. Atenção: também há figuras na resposta. \end{imagem*}

\begin{center}
  \begin{tabulary}{0.8\textwidth}{*{50}{L}}
    \hline \hline \\
      Fração da figura que deve ser pintada  &   Figura  &   Fração da figura que ficou sem pintar  \\
    \hline \\
      $\frac{5}{6}$  &   \includegraphics[width=30pt, keepaspectratio]{../../livro/media/cap2/secoes/pinte-dois-tercos.png}  &  \\
    \hline \\
      $\frac{3}{4}$  &   \includegraphics[width=30pt, keepaspectratio]{../../livro/media/cap2/secoes/pinte-tres-oitavos.png}  &  \\
    \hline \\
      $\frac{2}{5}$  &   \includegraphics[width=36pt, keepaspectratio]{../../livro/media/cap2/secoes/pinte-nove-decimos.png}  &  \\
    \hline \\
      $\frac{2}{3}$  &   \includegraphics[width=30pt, keepaspectratio]{../../livro/media/cap2/secoes/pinte-dois-tercos.png}  &  \\
    \hline \\
      $\frac{3}{8}$  &   \includegraphics[width=30pt, keepaspectratio]{../../livro/media/cap2/secoes/pinte-tres-oitavos.png}  &  \\
    \hline \\
      $\frac{9}{10}$  &   \includegraphics[width=36pt, keepaspectratio]{../../livro/media/cap2/secoes/pinte-nove-decimos.png}  &  \\
    \hline \\
  \end{tabulary}
\end{center}







\subsection{Atividade}







\begin{enumerate} [\quad a)] %s
  \item     Em cada um dos três copos idênticos a seguir, indique a fração da capacidade do copo que está com água.     \mbox{} \newline       
\end{enumerate} %s
\begin{imagem*}[breakable]{}{}   FIGURA GEOMÉTRICA - Figuras podem ser feitas juntas e incluir  os itens.      \includegraphics[width=180pt, keepaspectratio]{../../livro/media//cap2/secoes/copos-de-agua-01.png}   \end{imagem*}
\begin{enumerate} [\quad a)] %s
  \item     Qual é a fração da capacidade do copo correspondente à toda a água que está nos três copos?
  \item     É possível armazenar a água dos três copos em um único copo sem que transborde? Explique.
\end{enumerate} %s








\subsection{Atividade}







\begin{imagem*}[breakable]{}{}   FIGURA GEOMÉTRICA - Figuras devem ser feitas individualmente sem os itens ou textos. Observe que há figuras análogas na resposta\end{imagem*} 

\begin{center}
  \begin{tabulary}{0.8\textwidth}{*{50}{L}}
    \hline \hline \\
      Fração da unidade  &   Figura correspondente à fração da unidade  &   Desenhe aqui uma unidade  \\
    \hline \\
      $\frac{1}{2}$  &   \includegraphics[width=30pt, keepaspectratio]{../../livro/media/cap2/secoes/complete-retangulo.png}  &  \\
    \hline \\
      $\frac{4}{2}$  &   \includegraphics[width=30pt, keepaspectratio]{../../livro/media/cap2/secoes/complete-retangulo.png}  &  \\
    \hline \\
      $\frac{3}{2}$  &   \includegraphics[width=30pt, keepaspectratio]{../../livro/media/cap2/secoes/complete-retangulo.png}  &  \\
    \hline \\
      $\frac{2}{3}$  &   \includegraphics[width=30pt, keepaspectratio]{../../livro/media/cap2/secoes/complete-retangulo.png}  &  \\
    \hline \\
      $\frac{1}{2}$  &   \includegraphics[width=30pt, keepaspectratio]{../../livro/media/cap2/secoes/complete-disco.png}  &  \\
    \hline \\
      $\frac{4}{2}$  &   \includegraphics[width=30pt, keepaspectratio]{../../livro/media/cap2/secoes/complete-disco.png}  &  \\
    \hline \\
      $\frac{3}{2}$  &   \includegraphics[width=30pt, keepaspectratio]{../../livro/media/cap2/secoes/complete-disco.png}  &  \\
    \hline \\
      $\frac{2}{3}$  &   \includegraphics[width=30pt, keepaspectratio]{../../livro/media/cap2/secoes/complete-disco.png}  &  \\
    \hline \\
      $\frac{1}{2}$  &   \includegraphics[width=30pt, keepaspectratio]{../../livro/media/cap2/secoes/complete-segmento.png}  &  \\
    \hline \\
      $\frac{4}{2}$  &   \includegraphics[width=30pt, keepaspectratio]{../../livro/media/cap2/secoes/complete-segmento.png}  &  \\
    \hline \\
      $\frac{3}{2}$  &   \includegraphics[width=30pt, keepaspectratio]{../../livro/media/cap2/secoes/complete-segmento.png}  &  \\
    \hline \\
      $\frac{2}{3}$  &   \includegraphics[width=30pt, keepaspectratio]{../../livro/media/cap2/secoes/complete-segmento.png}  &  \\
    \hline \\
      $\frac{1}{2}$  &   \includegraphics[width=30pt, keepaspectratio]{../../livro/media/cap2/secoes/complete-hexagono.png}  &  \\
    \hline \\
      $\frac{4}{2}$  &   \includegraphics[width=30pt, keepaspectratio]{../../livro/media/cap2/secoes/complete-hexagono.png}  &  \\
    \hline \\
      $\frac{3}{2}$  &   \includegraphics[width=30pt, keepaspectratio]{../../livro/media/cap2/secoes/complete-hexagono.png}  &  \\
    \hline \\
      $\frac{2}{3}$  &   \includegraphics[width=30pt, keepaspectratio]{../../livro/media/cap2/secoes/complete-hexagono.png}  &  \\
    \hline \\
  \end{tabulary}
\end{center}







\subsection{Atividade}







Lucas, Matheus, Heitor, Rafael, Enzo, Nicolas, Lorenzo, Guilherme e Samuel estavam brincando de empurrar seus carrinhos de brinquedo para ver qual carrinho ia mais longe em uma pista reta.

A figura a seguir mostra o quão longe foi o carrinho de Lucas e onde ele parou na pista com relação ao ponto de largada.
\begin{imagem*}[breakable]{}{}   FIGURA ARTÍSTICA - deve incluir o texto. Entendendo bem a atividade pode tentar fazer algo mais bonito. Obseve que há figura similar na Resposta\end{imagem*}
\includegraphics[width=600pt, keepaspectratio]{../../livro/media//cap2/secoes/corrida-01.png}

Sabe-se que:

\begin{enumerate} [\quad a)] %s
  \item     O carrinho de Matheus só conseguiu ir metade da distância percorrida pelo carrinho de Lucas.
  \item     O carrinho de Heitor conseguiu ir até     $\frac{3}{2}$     da distância percorrida pelo carrinho de Lucas. 
  \item     O carrinho de Rafael conseguiu ir até     $\frac{4}{2}$     da distância percorrida pelo carrinho de Lucas.
  \item     O carrinho de Enzo conseguiu ir até     $\frac{5}{2}$     da distância percorrida pelo carrinho de Lucas. 
  \item     O carrinho de Nicolas conseguiu ir até     $\frac{6}{2}$     da distância percorrida pelo carrinho de Lucas. 
  \item     O carrinho de Lorenzo conseguiu ir até     $\frac{6}{4}$     da distância percorrida pelo carrinho de Lucas. 
  \item     O carrinho de Guilherme conseguiu ir até o dobro da distância percorrida pelo carrinho de Lucas.
  \item     O carrinho de Samuel conseguiu ir até     $\frac{6}{3}$     da distância percorrida pelo carrinho de Lucas. 
\end{enumerate} %s


Com estas informações, marque as posições de parada dos carrinhos de todos os amigos de Lucas no encarte que você irá receber.

\includegraphics[width=600pt, keepaspectratio]{../../livro/media//cap2/secoes/corrida-01.png}

Os carrinhos de Rafael e Samuel pararam no mesmo lugar? Explique.















\section*{QUEBRANDO A CUCA }



\subsection{Atividade}







(NAEP, 1992) Pense cuidadosamente nesta questão. Escreva uma resposta completa. Você pode usar desenhos, palavras e números para explicar sua resposta. Certifique-se de mostrar todo o seu raciocínio.

José comeu $\frac{1}{2}$ de uma pizza. Ella comeu $\frac{1}{2}$ de uma outra pizza. José disse que ele comeu mais pizza do que Ella, mas Ella diz que eles comeram a mesma quantidade. Use palavras, figuras ou números para mostrar que José pode estar certo.






\subsection{Atividade}







Miguel disse para Alice que a parte pintada de vermelho na figura a seguir corresponde a $\frac{3}{5}$ da figura, pois ela está dividida em 5 partes e 3 partes estão pintadas. Você concorda com a afirmação e a justificativa de Miguel? Explique!
\begin{imagem*}[breakable]{}{}   FIGURA GEOMÉTRICA.  
    \includegraphics[width=120pt, keepaspectratio]{../../livro/media//cap2/secoes/sera-tres-quintos-no-quadrado.png}  
\end{imagem*}





\subsection{Atividade}







A figura a seguir tem 3 partes pintadas de vermelho e 4 partes pintadas de branco. É correto afirmar que a parte pintada de vermelho corresponde a $\frac{3}{4}$ da figura? Explique.
\begin{imagem*}[breakable]{}{}   FIGURA GEOMÉTRICA -   
    \includegraphics[width=180pt, keepaspectratio]{../../livro/media//cap2/secoes/tres-setimos-ou-tres-quartos.png}  
\end{imagem*}






\subsection{Atividade}







\begin{enumerate} [\quad a)] %s
  \item     A parte em vermelho na figura a seguir representa     $\frac{1}{2}$     ou     $\frac{1}{4}$    ? 
\end{enumerate} %s
\begin{imagem*}[breakable]{}{}   FIGURA GEOMÉTRICA.    \includegraphics[height=60pt, keepaspectratio]{../../livro/media//undefined/hexagonos-unidade-01.png}\end{imagem*}
\begin{enumerate} [\quad a)] %s
  \item     A parte em vermelho na figura a seguir representa     $\frac{1}{2}$     ou     $\frac{3}{2}$    ? 
\end{enumerate} %s
\begin{imagem*}[breakable]{}{}   FIGURA GEOMÉTRICA.    \includegraphics[height=60pt, keepaspectratio]{../../livro/media//undefined/hexagonos-unidade-02.png}\end{imagem*}
\begin{enumerate} [\quad a)] %s
  \item     A parte em vermelho na figura a seguir representa     $\frac{3}{5}$     ou     $3$    ?
\end{enumerate} %s
\begin{imagem*}[breakable]{}{}   FIGURA GEOMÉTRICA.    \includegraphics[height=60pt, keepaspectratio]{../../livro/media//undefined/hexagonos-unidade-03.png}\end{imagem*}







\subsection{Atividade}







Júlia, Davi e Laura estavam estudando a figura a seguir.
\begin{imagem*}[breakable]{}{}   FIGURA GEOMÉTRICA -  Também há figuras em Resposta.  
    \includegraphics[width=180pt, keepaspectratio]{../../livro/media//cap2/secoes/tres-quintos-ou-cinco-tercos.png}  
\end{imagem*}

Júlia disse: ``A parte em vermelho representa $\frac{3}{5}$''. Davi retrucou: ``Não, não! A parte em vermelho representa $\frac{3}{2}$!''. Laura, então acrescentou: ``Eu acho que a parte em vermelho representa $3$!''. Quem está certo? Júlia, Davi ou Laura? Explique!







\subsection{Atividade}






Em uma pizzaria rodízio, 7 amigos comem, ao todo, 38 fatias (alternativa: simplesmente fazer o desenho das 38 fatias alinhadas, e não formando o círculo, com um triângulo ao lado de outro, caso as fatias fossem triângulos).
\begin{imagem*}[breakable]{}{}   FIGURA ARTÍSTICA   
    \includegraphics[width=180pt, keepaspectratio]{../../livro/media//cap2/secoes/38_oitavos_de_pizza.jpg}  
\end{imagem*}

Sabendo que nessa pizzaria cada pizza é equiparticionada em 8 partes, pergunta-se: 
\begin{enumerate} [\quad a)] %s
  \item     Quantas pizzas inteiras comerarm os 7 amigos? 
  \item     Que fração de uma pizza comeram  ao todo os amigos? 
  \item     É possível que todos os amigos tenham comido o mesmo número de fatias de pizza? Explique.
\end{enumerate} %s







\end{document}