\documentclass[a4,12pt]{book}

\usepackage{template}

\begin{document}


\chapter{Frações e a reta numérica }

\section{EXPLORANDO O ASSUNTO }

\subsection{Atividade}

Os quadrinhos a seguir mostram uma caixa-d'água sendo enchida. 
Para saber que fração da capacidade da caixa-d'água já está com água, será usada uma faixa graduada para indicar o nível de água na caixa. 

\begin{imagem*}[breakable]{}{}  Figura Artística  
  
    \includegraphics[width=120pt, keepaspectratio]{/var/www/livro/data/gitrepo/media//cap3/secoes/caixa_modelo.png}  
  A figura deve ilustrar quatro quadrinhos, cada um contendo a mesma caixa acima desenhada. A primeira com um quarto da caixa cheia, a seguinte com   $\frac{1}{2}$   do nível máximo, a terceira com três quartos e, por último, a caixa-d'água cheia. Abaixo de cada quadrinho, indicar: momento 1, momento 2, momento 3 e momento 4.  
\end{imagem*}

Escolha, para cada um dos momentos, a graduação que lhe parece mais adequada para registrar a quantidade de agua representada em cada uma das imagens. Explique sua escolha.


\begin{imagem*}[breakable]{}{}  Figura Artística  
  
    \includegraphics[width=120pt, keepaspectratio]{/var/www/livro/data/gitrepo/media//cap3/secoes/caixa_graduada_modelo.png}  
  Apresentar a imagem da caixa-d'água acima 4 vezes, cada uma com uma das graduações mostradas abaixo. APESAR DO QUE ESTÁ NA FIGURA, ESTAS CAIXAS GRADUADAS DEVEM ESTAR VAZIAS (sem água).  
  Atenção:       
\begin{itemize} %s
    \item       as marcas do 0 e do 1 devem coincidir, respectivamente, com a base e o topo da caixa.
    \item       os números devem vir bem ao lado da marcação, exceto o       $\frac{1}{2}$       no item a), que está entre duas marcações. Por esse motivo a caixa está sobre uma bancada, para que a graduação possa sobrar embaixo e em cima da caixa.
\end{itemize} %s
  
    \includegraphics[width=300pt, keepaspectratio]{/var/www/livro/data/gitrepo/media//cap3/secoes/graduacoes_da_caixa.jpg}  
\end{imagem*}

\subsection{Atividade}

Relembrando a representação na reta numérica: Você já conhece a reta numérica com os números naturais destacados.

\begin{imagem*}[breakable]{}{}   FIGURA GEOMÉTRICA - Há figuras geométricas nas respostas também.  
  
  ILUSTRAÇÃO DA RETA NUMÉRICA COM OS NÚMEROS NATURAIS DESTACADOS - de 1 a 5  
\end{imagem*}

\begin{enumerate} [\quad a)] %s
  \item     Marque na reta numérica pontos que representem as quantidades de pizza nas imagens a seguir. 
\end{enumerate} %s


\mbox{} \newline  \begin{imagem*}[breakable]{}{}   FIGURA ARTÍSTICA   \mbox{} \newline    Ilustrações relativas a 1 pizza, 2 pizzas, 4 pizzas  e a imagem de uma reta numérica com marcações de 0 a 5 e submarcações correspondentes a metades, como na imagem:     \includegraphics[width=180pt, keepaspectratio]{/var/www/livro/data/gitrepo/media//cap3/secoes/reta_1_a_5.jpg}  
  
\end{imagem*}


\begin{enumerate} [\quad a)] %s
  \item     E no caso destas imagens, que pontos na reta numérica representam as quantidades de pizza ilustradas?
\end{enumerate} %s


\mbox{} \newline  \begin{imagem*}[breakable]{}{}   FIGURA ARTÍSTICA   \mbox{} \newline    Ilustrações relativas a 1/2 pizza,  1/4 pizza, 3/4 pizza, 1 e meia pizzas. OBS: No caso de 1/4 e 3/4 de pizza, as ilustrações têm que deixar clara a partição da pizza em 4, das quais 1 ou 3 partes estão destacadas   
  
  Essas imagens são seguidas de uma reta numérca em que estão destacados as marcações do 0, do 1 e do 2 e submarcações que identifiquem quartos, como na imagem:     \includegraphics[width=180pt, keepaspectratio]{/var/www/livro/data/gitrepo/media//cap3/secoes/reta_quartos.jpg}   \end{imagem*}

\subsection{Atividade}

Para cada par ou trio de figuras a seguir, há uma reta numérica. Considerando a região colorida como uma fração da figura, ligue CADA figura ao número, sobre a reta numérica, correspondente à região colorida da mesma.
\begin{imagem*}[breakable]{}{}   FIGURA GEOMÉTRICA - Em cada par ou trio, as figuras devem ser congruentes entre si: . Podem ser usadas cores diferentes em cada par ou trio. Também há figuras geométricas na Resposta.  
    \includegraphics[width=240pt, keepaspectratio]{/var/www/livro/data/gitrepo/media//cap3/secoes/retasgrad1.jpg}     
    \includegraphics[width=240pt, keepaspectratio]{/var/www/livro/data/gitrepo/media//cap3/secoes/retasgrad2.jpg}  
\end{imagem*}

\subsection{Atividade}

Para cada uma das figuras a seguir, marque na reta numérica o ponto correspondente à fração da unidade destacada na imagem:


( ERRO:\{<WRAP\} ) Imagem> Nas figuras abaixo incluir apenas imagens como as apresentadas (sem a descrição), seguidas, cada uma, de uma reta numérica como descrito:
( ERRO:\{</WRAP\} )>

\begin{enumerate} [\quad a)] %s
  \item     A unidade é uma pizza. 
\end{enumerate} %s
\begin{imagem*}[breakable]{}{}   FIGURA ARTÍSTICA e GEOMÉTRICA   \mbox{} \newline     5/8 de uma pizza   \mbox{} \newline        \includegraphics[width=60pt, keepaspectratio]{/var/www/livro/data/gitrepo/media//cap3/secoes/cinco_oitavos_de_pizza.png}     \mbox{} \newline    Reta 1: 0, 3/8, 5/8, 1 \end{imagem*}
\begin{enumerate} [\quad a)] %s
  \item     A unidade é uma barra de chocolate 
\end{enumerate} %s
\begin{imagem*}[breakable]{}{}   FIGURA ARTÍSTICA   \mbox{} \newline    3/8 de uma barra de chocolate;   \mbox{} \newline    Incluir uma imagem de barra de chocoolate destacando oitavos, dos quais três não aparecem e   \mbox{} \newline    Reta 2: 0, 3/8, 5/8, 1 \end{imagem*}
\begin{enumerate} [\quad a)] %s
  \item     A unidade é uma maçã. 
\end{enumerate} %s
\begin{imagem*}[breakable]{}{}   FIGURA ARTÍSTICA  e GEOMÉTRICA - importante que a visualização da maçã deixe claro que trata-se de meia maçã.   \mbox{} \newline    1/2 de uma maçã     \includegraphics[width=60pt, keepaspectratio]{/var/www/livro/data/gitrepo/media//cap3/secoes/meia_maca.jpg}     \mbox{} \newline    Reta 3: 0, 1/4, 1/2, 3/4, 1 \end{imagem*}
\begin{enumerate} [\quad a)] %s
  \item     A unidade é um sanduíche de queijo com presunto. 
\end{enumerate} %s
\begin{imagem*}[breakable]{}{}   FIGURA ARTÍSTICA  e GEOMÉTRICA   \mbox{} \newline    1/4 de um  sanduiche de queijo com presunto   \mbox{} \newline        \includegraphics[width=120pt, keepaspectratio]{/var/www/livro/data/gitrepo/media//cap3/secoes/um_quarto_sanduiche.png}     \mbox{} \newline    Reta 4: 0, 1/4, 1/2, 3/4, 1 \end{imagem*}
\begin{enumerate} [\quad a)] %s
  \item     A unidade é uma torta. 
\end{enumerate} %s
\begin{imagem*}[breakable]{}{}   FIGURA ARTÍSTICA  e GEOMÉTRICA  \mbox{} \newline    Uma  torta inteira (mas fatiada)   \mbox{} \newline        \includegraphics[width=120pt, keepaspectratio]{/var/www/livro/data/gitrepo/media//cap3/secoes/bolo_fatiado.jpg}     \mbox{} \newline    Reta 5: 0, 1/2, 1, 3/2, 2 e 5/2 \end{imagem*}
\begin{enumerate} [\quad a)] %s
  \item     A unidade é um biscoito. 
\end{enumerate} %s
\begin{imagem*}[breakable]{}{}   FIGURA ARTÍSTICA  e GEOMÉTRICA   \mbox{} \newline    Dois biscoitos e meio   \mbox{} \newline        \includegraphics[width=60pt, keepaspectratio]{/var/www/livro/data/gitrepo/media//cap3/secoes/dois_biscoitos_e_meio.png}     \mbox{} \newline    Reta 6: 0, 1/2, 1, 3/2, 2 e 5/2 \end{imagem*}
\begin{enumerate} [\quad a)] %s
  \item     A unidade é um copo cheio. 
\end{enumerate} %s
\begin{imagem*}[breakable]{}{}   FIGURA ARTÍSTICA e GEOMÉTRICA  \mbox{} \newline     copo vazio   \mbox{} \newline        \includegraphics[width=60pt, keepaspectratio]{/var/www/livro/data/gitrepo/media//cap3/secoes/copo.jpg}     \mbox{} \newline    Reta 7: 0, 1/2, 1, 3/2, 2 e 5/2 \end{imagem*}

\subsection{Atividade}

A faixa a seguir está dividida em 5 partes iguais. 
\begin{imagem*}[breakable]{}{}   FIGURAS GEOMÉTRICAS - Há figuas na resposta também.  
    \includegraphics[width=180pt, keepaspectratio]{/var/www/livro/data/gitrepo/media//undefined/unidade_em_quintos.png}  
\end{imagem*}

\begin{enumerate} [\quad a)] %s
  \item     Considerando a faixa como unidade, escreva na reta numérica a fração correspondente a cada uma das regiões coloridas.
\end{enumerate} %s

\includegraphics[width=180pt, keepaspectratio]{/var/www/livro/data/gitrepo/media//cap3/secoes/quintos.png} \mbox{} \newline  

\begin{enumerate} [\quad a)] %s
  \item     Escreva, em linguagem simbólica, a fração correspondente à faixa inteira. De que outra maneira é possível indicar essa quantidade? 
\end{enumerate} %s

\begin{enumerate} [\quad a)] %s
  \item         \mbox{} \newline      
\end{enumerate} %s
( ERRO:\{<WRAP\} ) IMAGEM> FIGURAS GEOMÉTRICAS - Incluir imagem da reta que aparece na figura \includegraphics[width=180pt, keepaspectratio]{/var/www/livro/data/gitrepo/media//cap3/secoes/quintos.png} completa com as indicações das frações $\frac{1}{5}$, $\frac{2}{5}$, $\frac{3}{5}$ e $\frac{4}{5}$ ( ERRO:\{</WRAP\} )>
\begin{enumerate} [\quad a)] %s
  \item     A faixa inteira é igual a cinco quintos. Esta fração pode ser representada simbolicamente como     $\frac{5}{5}$    . A fração     $\frac{5}{5}$     da barra é igual a uma barra inteira, isto é,     $\frac{5}{5}=1$    . 
\end{enumerate} %s


\subsection{Atividade}

A professora Julia pediu que os seus alunos, Pedro e Miguel, marcassem $\frac{1}{2}$ na reta numérica traçada em uma fita, como esta que vocês também receberam:

\begin{imagem*}[breakable]{}{}   FIGURA GEOMÉTRICA - Há figura no   ``Para o professor''   também.  
    \includegraphics[width=240pt, keepaspectratio]{/var/www/livro/data/gitrepo/media//cap3/secoes/fita_1_rodrigo.jpg}  
\end{imagem*}

Pedro trouxe a seguinte marcação 
\begin{imagem*}[breakable]{}{}   FIGURA GEOMÉTRICA - Há figura no   ``Para o professor''   também.  
    \includegraphics[width=240pt, keepaspectratio]{/var/www/livro/data/gitrepo/media//cap3/secoes/fita_3.jpg}  
\end{imagem*}

Miguel trouxe esta
\begin{imagem*}[breakable]{}{}   FIGURA GEOMÉTRICA - Há figura no   ``Para o professor''   também.  
    \includegraphics[width=240pt, keepaspectratio]{/var/www/livro/data/gitrepo/media//cap3/secoes/fita_2.jpg}  
\end{imagem*}

\begin{enumerate} [\quad a)] %s
  \item     É possível ambos estarem corretos? Justifique sua resposta. 
  \item     Faça marcações nessa fita correspondentes a     $\frac{1}{4}$     e a     $\frac{3}{4}$    . Explique como você fez essas marcações.
  \item     Onde deve ser feita a marcação correspondente a     $\frac{4}{4}$    ?
  \item     E a marcação de     $\frac{5}{4}$    ? 
\end{enumerate} %s

\subsection{Atividade}

Um caçador de tesouros encontrou o mapa a seguir. Leia as instruções para a localização do tesouro e decida em que local ele deve cavar:

( ERRO:\{<WRAP\} ) Imagem> FIGURA ARTÍSTICA -

Incluir imagem de um mapa de tesouro que ocupe cerca de meia folha tamanho A4 

\subsection{Atividade}

Três amigos foram a uma pizzaria e cada um pediu uma pizza média, de três sabores diferentes: João comeu $\frac{3}{4}$ da pizza de calabresa, Maria comeu  $\frac{2}{4}$ da pizza de presunto e Miguel comeu $\frac{3}{5}$ da pizza de Milho. Sabendo que todas as pizzas eram do mesmo tamanho, pergunta-se:
\begin{enumerate} [\quad a)] %s
  \item     Quem comeu mais pizza, João ou Maria? Explique.
  \item     E no caso de João e Miguel, quem comeu mais pizza? Explique.
  \item     Dos três amigos, quem comeu mais pizza? Explique.
  \item     Marque na reta numérica a seguir as frações correspondentes às porções de pizza que cada amigo comeu, e confirme na reta numérica sua resposta em c.
\end{enumerate} %s

\begin{imagem*}[breakable]{}{}   FIGURA GEOMÉTRICA  
  
  Desenhar uma reta numérica onde estão marcados o 0 e o 1. O tamanho pode ser 12 cm, com 0 e 1 próximos às extremidades.  
\end{imagem*}

\subsection{Atividade}

A imagem a seguir ilustra uma tartaruga percorrendo um caminho em linha reta, do ponto de partida ao de chegada. Observe a posição da tartaruga na imagem e avalie se as afirmações a seguir estão corretas ou não. Em cada item, explique a sua avaliação por escrito.
\begin{enumerate} [\quad a)] %s
  \item     A tartaruga percorreu mais do que a metade do percurso total.
  \item     A tartaruga percorreu mais do que     $\frac{3}{4}$     do percurso total.
  \item     A tartaruga percorreu mais do que     $\frac{3}{8}$     do percurso total.
  \item     A tartaruga percorreu menos do que     $\frac{3}{4}$     do percurso total.
  \item     A tartaruga percorreu menos do que     $\frac{2}{8}$     do percurso total.
  \item     A tartaruga percorreu menos do que     $\frac{2}{3}$     do percurso total.
  \item     A tartaruga percorreu     $\frac{3}{4}$     do percurso total.
  \item     A tartaruga percorreu pelo menos     $\frac{5}{8}$     do percurso total.
  \item     Para alcançar a chegada, a tartaruga precisa percorrer mais do que a metade do caminho.
  \item     Para alcançar a chegada, a tartaruga precisa percorrer menos do que     $\frac{2}{3}$     do caminho.
\end{enumerate} %s


\begin{imagem*}[breakable]{}{}   FIGURAS ARTÍSTICAS - O percurso deve estar dividido em 24 partes iguais, como ilustra a imagem. A tartaruga deve estar na 11a marcação. Atenção: também há figuras em   ``Resposta''  .  
  
    \includegraphics[width=\textwidth,height=4cm, keepaspectratio]{/var/www/livro/data/gitrepo/media//cap3/secoes/tartaruga.jpg}  
\end{imagem*}

\subsection{Atividade}

Na figura, há várias retas paralelas igualmente espaçadas e outra reta, destacada em vermelho, não paralela às anteriores. Observe que as retas paralelas marcam na reta destacada em vermelho pontos também igualmente espaçados. Dois desses pontos correspondem ao 0 e ao 1. A reta vermelha torna-se uma reta numérica, como ilustra a figura. 

\begin{enumerate} [\quad a)] %s
  \item     Marque, usando os pontos destacados na reta numérica, a fração     $\frac{1}{2}$    . 
  \item     Associe frações a cada um dos pontos destacados na reta numérica. Explique a sua resposta.    \mbox{} \newline      
\end{enumerate} %s
( ERRO:\{<WRAP\} ) Imagem> FIGURA GEOMÉTRICA -  A linha transversal ao feixe de paralelas (igualmente espaçadas) deve estar colorida em vermelho. Devem estar destacados nessa reta, além do 0 e do 1, 9 pontos, como ilustra a figura. Uma segunda imagem, semelhante a esta, precisa ser inserida no espaço de resposta. Ver orientações. \mbox{} \newline  \includegraphics[width=180pt, keepaspectratio]{/var/www/livro/data/gitrepo/media//cap3/secoes/parelas.jpg} ( ERRO:\{</WRAP\} )>\mbox{} \newline  \mbox{} \newline  Como na figura anterior, há várias retas paralelas igualmente espaçadas e outra reta, destacada em azul, não paralela às anteriores. Observe que as retas paralelas marcam na reta destacada em azul pontos também igualmente espaçados. Dois desses pontos correspondem às frações $\frac{1}{2}$ e $\frac{3}{2}$, como ilustra a figura. 
\begin{enumerate} [\quad a)] %s
  \item     Marque, usando os pontos destacados na reta numérica, os pontos correspondentes ao 0 e ao 1
  \item     Marque, nesta mesma reta numérica, as frações     $\frac{3}{4}$     e     $\frac{5}{4}$    .
\end{enumerate} %s


( ERRO:\{<WRAP\} ) Imagem> FIGURA GEOMÉTRICA -  A linha transversal ao feixe de paralelas (igualmente espaçadas) deve estar colorida em vermelho. Devem estar destacados nessa reta, além das frações $\frac{1}{2}$ e $\frac{3}{2}$, outros 8 pontos, como ilustra a figura.

Uma segunda imagem, semelhante a esta, precisa ser inserida no espaço de resposta. Ver orientações.

\includegraphics[width=180pt, keepaspectratio]{/var/www/livro/data/gitrepo/media//cap3/secoes/reta_meios.jpg}

( ERRO:\{</WRAP\} )>



\section{ORGANIZANDO AS IDEIAS }


Frações na reta numérica \mbox{} \newline 


Já é conhecido que os números naturais podem ser representados por pontos em uma reta. 
Para isso, é preciso começar escolhendo dois pontos que vão corresponder ao 0 e ao 1 e, a partir deles, são marcados os pontos que corresponderão aos demais números naturais.

\begin{imagem*}[breakable]{}{}   FIGURAS GEOMÉTRICAS -   {\bf As imagens das retas têm que ser congruentes, os segmentos 0-1 em todas elas tê que ser congruentes. Não podem ter dimensões diferentes como nas imagens ilustrativas. }     \mbox{} \newline   
  A determinação dos pontos correspondentes ao  2, 3, 4, 5, e assim por diante, devem ser ilustradas pelo uso do compasso, sustentando a justaposição de segmentos congruentes, como nas imagens.   
    \includegraphics[width=240pt, keepaspectratio]{/var/www/livro/data/gitrepo/media//cap3/secoes/reta_numerica_unidade.png}    \end{imagem*}
\begin{imagem*}[breakable]{}{}       \includegraphics[width=300pt, keepaspectratio]{/var/www/livro/data/gitrepo/media//cap3/secoes/reta012.png}    \end{imagem*}
\begin{imagem*}[breakable]{}{}       \includegraphics[width=510pt, keepaspectratio]{/var/www/livro/data/gitrepo/media//cap3/secoes/reta0127.png}   \end{imagem*}

As frações também podem ser associadas a pontos na reta numérica. Para isso, é preciso identificar o segmento unitário, aquele cujos extremos são os pontos correspondentes ao 0 e ao 1. Esse segmento representa a unidade.

\begin{imagem*}[breakable]{}{}   MESMA IMAGEM DAS RETAS ANTERIORES, MAS APENAS COM O SEGMENTO UNITÁRIO (DE EXTREMIDADES 0 E 1) EM DESTAQUE  \end{imagem*}

Dividindo a unidade em partes iguais,  cada uma das partes identifica uma fração da unidade na reta numérica.
Por exemplo, a divisão da unidade em 3 partes iguais identifica terços. O ponto correspondente a $\frac{1}{3}$  é a extremidade do segmento que, a partir do 0, identifica o primeiro terço da unidade. 

\begin{imagem*}[breakable]{}{}       \includegraphics[width=240pt, keepaspectratio]{/var/www/livro/data/gitrepo/media//cap3/secoes/reta_numerica_umterco.png}   \end{imagem*}

A partir dele, por justaposições desse segmento, são identificados na reta numérica os pontos correspondentes a $\frac{2}{3}$,$\frac{3}{3}$,$\frac{4}{3}$, e assim por diante. 

\begin{imagem*}[breakable]{}{}       \includegraphics[width=240pt, keepaspectratio]{/var/www/livro/data/gitrepo/media//cap3/secoes/reta_numerica_doistercos.png}    \end{imagem*}
\begin{imagem*}[breakable]{}{}       \includegraphics[width=360pt, keepaspectratio]{/var/www/livro/data/gitrepo/media//cap3/secoes/reta_numerica_tercos.png}   \end{imagem*}

A representação dos números na reta numérica ajuda a perceber que os pontos correspondentes a algumas frações  são os mesmos que os correspondentes a alguns números naturais. Por exemplo, $\frac{3}{3}$ é igual a 1. 

\begin{imagem*}[breakable]{}{}   FIGURAS ARTÍSTICAS (2) E GEOMÉTRICAS (2), as duas primeiras lado a lado, a reta abaixo e a igualdade abaixo da reta:  
  
\begin{enumerate} [\quad a)] %s
    \item       UMA PIZZA PARTIDA EM 3/3, 
    \item       UMA BARRA DE CHOCOLATE PARTIDA EM 3/3,
    \item       RETA NUMÉRICA DESTACANDO A JUSTAPOSICAO DE TRÊS SEGMENTOS CORRESPONDENTES A 1/3 DA UNIDADE
    \item             $\frac{3}{3}=1$        
\end{enumerate} %s
  
\end{imagem*}

Já $\frac{6}{3}$ é igual a 2. 

\begin{imagem*}[breakable]{}{}   FIGURAS ARTÍSTICAS (2) E GEOMÉTRICAS (2), as duas primeiras lado a lado, a reta abaixo e a igualdade abaixo da reta:  
\begin{enumerate} [\quad a)] %s
    \item       DUAS PIZZA PARTIDAS EM TERÇOS,
    \item       DUAS BARRAS DE CHOCOLATE PARTIDAS EM TERÇOS,
    \item       RETA NUMÉRICA DESTACANDO A JUSTAPOSICAO DE SEIS SEGMENTOS CORRESPONDENTES A 1/3 DA UNIDADE
    \item             $\frac{6}{3}=2$        
\end{enumerate} %s
  
\end{imagem*}

E $\frac{12}{3}$, é igual a que número natural? $\frac{12}{3}=$ ( ERRO:\{\_\_\} )( ERRO:\{\_\_\} )( ERRO:\{\_\_\} )( ERRO:\{\_\_\} ) \mbox{} \newline 

Para identificar na reta numérica os pontos correspondentes às frações $\frac{1}{4}$, $\frac{2}{4}$, $\frac{3}{4}$, $\frac{4}{4}$,$\frac{5}{4}$, $\frac{6}{4}$, e assim por diante, o processo é o mesmo:

\begin{imagem*}[breakable]{}{}   FIGURA GEOMÉTRICA  
  
    \includegraphics[width=360pt, keepaspectratio]{/var/www/livro/data/gitrepo/media//cap3/secoes/reta_numerica_quartos.png}   \end{imagem*}

Na reta numérica a seguir estão destacados alguns pontos e as frações correspondentes a eles. Observe e complete as frações em destaque.

\begin{imagem*}[breakable]{}{}    FIGURA GEOMÉTRICA - IMAGEM DA RETA NUMERICA COM SEXTOS DESTACADOS:  
  
    \includegraphics[width=120pt, keepaspectratio]{/var/www/livro/data/gitrepo/media//cap3/secoes/reta_sextos.jpg}  
  
\end{imagem*}

A ordem na reta numérica\mbox{} \newline 

Na reta numérica, os números são organizados em ordem crescente, a partir do zero no sentido do 1. Assim, o que vale para o 0, o 1, o 2, o, 3, etc. também valerá para as frações: 

\begin{imagem*}[breakable]{}{}   FIGURA GEOMÉTRICA - INCLUIR RETA COM metades, quintos e décimos destacados \end{imagem*}

Na reta numérica, quanto mais distante do 0 estiver o ponto correspondente ao número, maior será o número. 

\begin{imagem*}[breakable]{}{}    FIGURA GEOMÉTRICA - IMAGEM DA RETA NUMÉRICA destacando os pontos correspondentes ao 4/5 e 4/3:  
  
\end{imagem*}

$\frac{4}{3}$ é maior do que $\frac{4}{5}$. Ou ainda, $\frac{4}{5}$ é menor do que $\frac{4}{3}$.


O símbolo é usado $<$ para dizer ``menor do que''.

Por exemplo, a frase ``oito é menor do que quinze'' pode ser expressa de modo mais resumido com ``$8<15$''. Já a expressão $\frac{1}{2}<\frac{3}{2}$ indica que ``um meio é menor do que três meios''.

Do mesmo modo, o símbolo $>$ é usado para significar ``maior do que'', portanto, também pode-se escrever $15>8$ para expressar ``quinze é maior do que oito'' ou $\frac{3}{2}>\frac{1}{2}$  para expressar ``três meios é maior do que um meio''

\begin{imagem*}[breakable]{}{}   FIGURA ARTÍSTICA  
  
    \includegraphics[width=300pt, keepaspectratio]{/var/www/livro/data/gitrepo/media//cap3/secoes/simbolos_maior_que_igual.jpg}   \end{imagem*}

\section{MÃO NA MASSA }

\subsection{Atividade}

Na reta numérica já estão marcados o 0, o 1 e a fração $\frac{1}{2}$. Marque $\frac{3}{2}$, $\frac{3}{4}$, $\frac{5}{4}$, $\frac{8}{4}$, $\frac{10}{4}$, $\frac{1}{8}$, $\frac{7}{8}$, $\frac{10}{8}$ e 2.
\begin{imagem*}[breakable]{}{}   FIGURA GEOMÉTRICA  
  
    \includegraphics[width=240pt, keepaspectratio]{/var/www/livro/data/gitrepo/media//cap3/secoes/reta.png}  
\end{imagem*}

\subsection{Atividade}

Na reta numérica a seguir estão destacados os pontos correspondentes ao 0, ao 1 e a $\frac{1}{2}$. Os demais pontos correspondem às frações apresentadas a seguir. Associe cada fração ao ponto correspondente.  

( ERRO:\{<WRAP\} ) Imagem> FIGURAS GEOMÉTRICAS - Há figura na resposta também.
Os pontos em destaque na reta numérica desta atividade têm que corresponder corretamente aas frações da unidade em destaque.
\includegraphics[width=360pt, keepaspectratio]{/var/www/livro/data/gitrepo/media//cap3/secoes/reta_e_fracoes.png}
( ERRO:\{</WRAP\} )>

\section{QUEBRANDO A CUCA }


\subsection{Atividade}

Observe as faixas pintadas na atividade anterior. 

\begin{enumerate} [\quad a)] %s
  \item     Corte uma das 3 partes da faixa verde. Que fração representa? Marque esta fração na reta numerica abaixo.
  \item     Corte agora 2 partes da faixa dividida em 6 partes e que foi pintada de rosa claro. Que fração da unidade estas duas partes cortadas representam? Marque esta fração na mesma reta numerica abaixo.
  \item     Corte agora 3 partes da faixa dividida em 9 partes e que foi pintada de vermelho. Que fração da unidade estas três partes cortadas representam? Marque esta fração na mesma reta numérica
\end{enumerate} %s


\begin{imagem*}[breakable]{}{}   FIGURA GEOMÉTRICA  
  
    \includegraphics[width=180pt, keepaspectratio]{/var/www/livro/data/gitrepo/media//cap3/secoes/reta.jpg}  .  
  
\end{imagem*}

Como ficou a marcação? O que podemos inferir do resultado obtido nos itens a), b) e c)?

Identifique na figura com as faixas da atividade anterior outras frações que podem ser marcadas no mesmo local na reta.


\subsection{Atividade}


Na reta numérica a seguir:
\begin{enumerate} [\quad a)] %s
  \item     Marque     $\frac{1}{2}$    . Justifique sua resposta.
  \item     Marque     $\frac{1}{4}$    ,     $\frac{3}{4}$     e     $\frac{5}{4}$    . Explique como raciocinou para fazer essas marcações. 
\end{enumerate} %s


\begin{imagem*}[breakable]{}{}   FIGURA GEOMÉTRICA  
    \includegraphics[width=240pt, keepaspectratio]{/var/www/livro/data/gitrepo/media//cap3/secoes/reta_4_partes_a.jpg}  
\end{imagem*}

Observando a reta numérica com as marcações feitas, compare: 
\begin{enumerate} [\quad a)] %s
  \item         $\frac{1}{4}$     é maior ou menor do que     $\frac{1}{2}$    ? E     $\frac{3}{4}$    , é maior ou menor do que      $\frac{1}{2}$    ?
  \item     Ao marcar o     $\frac{5}{4}$     o que podemos afirmar sobre a comparação com a unidade 1?
  \item     Escreva em ordem crescente as frações dadas. Será que podemos inferir alguma regra especial para ordenar frações?
\end{enumerate} %s


\subsection{Atividade}

Frações: a melhor escolha para dividir uma unidade

Na figura abaixo temos faixas divididas em 3, 4 e 5 partes. 

\begin{imagem*}[breakable]{}{}   FIGURAS GEOMÉTRICAS  
    \includegraphics[width=\textwidth,height=4cm, keepaspectratio]{/var/www/livro/data/gitrepo/media//cap3/secoes/figura_iniciaal.jpg}  
\end{imagem*}

Compare cada uma delas com a parte que sobrou na atividade anterior.  
Qual a melhor subdivisão nesse caso?
Explique com suas palavras a sua escolha.

Observe as faixas abaixo e em cada caso dê uma representação fracionária relacionada a cada uma das faixas, observando a subdivisão proposta.

\includegraphics[width=\textwidth,height=4cm, keepaspectratio]{/var/www/livro/data/gitrepo/media//cap3/secoes/faixas_ate_2.png}

\subsection{Atividade}

Frações: comparando frações unitárias

Na figura abaixo temos partes pintadas de três faixas:
\includegraphics[width=180pt, keepaspectratio]{/var/www/livro/data/gitrepo/media//cap3/secoes/3_faixas.png}

Para cada uma encontre as frações correspondentes e construa uma reta numérica e exiba uma localização aproximada para cada uma das frações.

\end{document}