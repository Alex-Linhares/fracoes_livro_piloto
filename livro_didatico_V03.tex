\documentclass[10 pt]{article}


\usepackage[latin1]{inputenc}

\usepackage[brazil]{babel}
\usepackage{enumerate}
\usepackage[autostyle]{csquotes}
\usepackage{hyperref}% to use \url{..} or \hyperref[label_name]{''link text''}
\renewcommand{\labelenumi}{\alph{enumi})}
\usepackage[autostyle]{csquotes}
%\input{unirioprojeto.tex}


\addtolength{\textwidth}{100pt} \addtolength{\textheight}{58pt}
\addtolength{\hoffset}{-50pt} \addtolength{\voffset}{-15pt}



%%%%%%%%%%%%%%%%%%%%%%%%%%%%%%%%%%%%%%%%%%%%%%%%%%%%%%%%%%%%%%%%%%%%%%%%
\providecommand{\U}[1]{\protect\rule{.1in}{.1in}}
%EndMSIPreambleData
\setlength{\topmargin}{-0.9 in} \setlength{\textwidth}{6.50 in}
%\setlength{\oddsidemargin}{0.2 in} \setlength{\evensidemargin}{0.2 in}
\setlength{\textheight}{9.5 in} \setlength{\marginparwidth}{0.5 in}
\setlength{\marginparsep}{0.5 in}
\renewcommand{\baselinestretch}{1.5}



\begin{document}

%    \titulo{Unirio de portas abertas}{2.2010}
%\includegraphics[scale=0.25]{unirio.png} %\includegraphics[scale=0.7]{faperj}

\begin{center}
{\large{\bf   Livro Aberto de Matemática}}
 \end{center}


    \section{Identificação do projeto}
\textbf{T\'itulo:} Livro Aberto de Matemática

\begin{flushright}
\noindent\begin{tabular}{llll}
 &&\\
 &\textbf{Professor coordenador:}& Fábio Luiz Borges Simas (UNIRIO)\\
 &\textbf{Pesquisadores associados:}& Augusto Quadros Teixeira (IMPA)
 
    \end{tabular}
          
\end{flushright}	
        \vspace{0.3cm}

\section{Resumo}

Este é um esforço de professores da Educação Básica e Superior, assim como de entusiastas, para produzir, até o final de 2017, uma coleção de livros didáticos de Matemática com código aberto, para o segundo ciclo do Ensino Fundamental, nos moldes do Plano Nacional do Livro Didático (PNLD).
A esta obra será atribuída a licença {\it Creative Commons} (CC) BY 4.0, isto significa que o material poderá ser livremente distribuído e alterado, mesmo que para fins comerciais.
Uma equipe vinculada ao projeto coordenará os esforços dessa tarefa.

O conteúdo dessa obra proverá de diversas fontes, após ser avaliada e adaptada pelo corpo de coordenadores.
Essas fontes incluem: contribuições da própria equipe de coordenadores, adaptações de Trabalhos de Conclusão de Curso (TCC) de graduação e mestrado, contribuições de professores e entusiastas e finalmente trabalhos disponibilizados com licenças compatíveis.

Espera-se que outros colaboradores ajudem na elaboração do material, aplicando os recursos em suas aulas, com comentários e até mesmo editando o material.
Esta cooperação se dará através de um site explicando a filosofia do projeto, onde será possível visualizar, baixar e comentar o texto já produzido.
Após cadastrar-se, o visitante também poderá editar o texto no próprio site, assim como na Wikipédia.
Com a diferença de que neste projeto todas as edições serão avaliadas pela equipe de revisão e a autoria de cada trecho do material será atribuída ao colaborador que a escreveu, desde que seja mantida a licença aberta do material.

\section{Relevância da proposta}

Ao ser construída com a filosofia {\it open source}, esta obra adquire um caráter permanente, de modo que independente dos organizadores, qualquer pessoa ou editora pode apropriar-se total ou parcialmente dos recursos nela disponíveis desde que respeitando os termos da licença %(\url{http://creativecommons.org/licenses/by/4.0/deed.pt_BR}).

O preço de venda dos livros desta coleção não pode ser excessivo porque qualquer um pode editar e imprimir o mesmo material disponível no site.
Acredita-se que a ``Educação Aberta'' tenha potencial para readequar os preços dos livros didáticos brasileiros em um nível mais baixo.

Além da redução nos custos dos livros didático, esse projeto também visa melhorar a qualidade desses materiais.
Isso se dará através dos trabalhos de editoração e revisão feitos pela equipe coordenadora, assim como pela contribuição plural de diversos profissionais e entusiastas com formações e habilidades complementares.

Apesar de toda a beleza desta iniciativa, de pouco ela vale se os professores das escolas decidirem não usufruir dos recursos disponíveis em nossos livros.
Neste sentido será fundamental a popularização e a divulgação massiva do material entre os professores das escolas ainda na fase de elaboração.
Na medida em que ele se torna autor ou colaborador do material didático, torna-se natural que ele o escolha para trabalhar com seus alunos.
Além disso, espera-se que o professor se coloque numa postura mais criativa no processo de ensino em contraposição ao lugar de mero reprodutor do conteúdo na ordem e do modo que são apresentados no livro didático escolhido.


\section{Introdução}

% O livro didático na maioria das vezes é a principal ferramenta utilizada pelos professores da escola para prepararem as suas aulas.
% Contudo
% 
% A experiência dos programas Open Source.
% 
% Falar do OER Commons
% 
% Preço e qualidade dos livros didáticos aprovados no PNLD.
% (artigo do Nilson José Machado Em Aberto, Brasília, ano 16, n.69, jan./mar.
% 1996)
% 
% Escola e professores podem adaptar o texto para seus estudantes.
% Professor deve se sentir autor do material para que dele se aproprie.
% BNC e
 \blockquote{``A forma básica de utilização [do livro didático], no entanto, foi mantida, e o livro `adotado' pelo professor - consumível ou não - praticamente determina o conteúdo a ser ensinado.
 O professor abdica do privilégio de projetar os caminhos a serem trilhados, em consonância com as circunstâncias - experiências, interesses, perspectivas - de seus alunos, passando a conformar-se, mais ou menos acriticamente, com o encadeamento de temas propostos pelo autor.
 Tal encadeamento ora tem características idiossincráticas, ora resulta da cristalização de certos percursos, que de tanto serem repetidos, adquirem certa aparência de necessidade lógica; nos dois casos, a passividade do professor torna um pouco mais difícil a já complexa tarefa da construção da autonomia intelectual dos alunos.''}
 (artigo do Nilson José Machado Em Aberto, Brasília, ano 16, n.69, jan./mar.
 1996 página 31).
% 
% Falar das áreas do BNC (Geometria, Grandezas e Medidas, etc)

O BNC determina alguns objetivos para o Ensino de Matemática na Educação Básica e mais especificamente no Ensino Fundamental.
 
\blockquote{``OBJETIVOS GERAIS DA ÁREA DE MATEMÁTICA NO ENSINO FUNDAMENTAL \label{objetivosBNC}
\begin{itemize}
 \item Identificar os conhecimentos matemáticos como meios para compreender o mundo à sua volta.
 \item Desenvolver o interesse, a curiosidade, o espírito de investigação e a capacidade para criar/elaborar e resolver problemas.
 \item Fazer observações sistemáticas de aspectos quantitativos e qualitativos presentes nas práticas sociais e culturais, sabendo selecionar, organizar e produzir informações relevantes, para interpretá-las e avaliá-las criticamente.
 \item Estabelecer relações entre conceitos matemáticos de um mesmo eixo e entre os diferentes eixos (Geometria, Grandezas e Medidas, Estatística e Probabilidade, Números e Operações, Álgebra e Funções), bem como entre a Matemática e outras áreas do conhecimento.
 \item Comunicar-se matematicamente (interpretar, descrever, representar e argumentar), fazendo uso de diferentes linguagens e estabelecendo relações entre ela e diferentes representações matemáticas.
 \item Desenvolver a autoestima e a perseverança na busca de soluções, trabalhando coletivamente, respeitando o modo de pensar dos/as colegas e aprendendo com eles/as.
 \item Recorrer às tecnologias digitais a fim de compreender e verificar conceitos matemáticos nas práticas sociocientíficas.''    
\end{itemize}}{(BNCC p. XX)}
    
% 
% Acreditamos que a democratização da educação passa pelo Open Source.
% 
% Potencial para uso em outras disciplinas escolares.


\section{Objetivos}

\begin{enumerate}
\item Produzir uma coleção de livros didáticos de matemática com código aberto e de livre distribuição e edição contendo quatro volumes para os estudantes e quatro manuais de professores nos moldes do PNLD.
\item Melhorar a qualidade dos livros didáticos de matemática utilizados nas escolas públicas do segundo ciclo do Ensino Fundamental.
\item Reduzir os preços das coleções de livros de matemática concorrentes no PNLD para o segundo ciclo do Ensino Fundamental.
\item Conseguir adesão de diversos professores das Escolas da Educação Básica para a ideia de {\it ``Educação Aberta''} e para esta coleção.
\item Estabelecer uma metodologia e plataforma de cooperação para produção de material didático que possa ser reproduzida em outras disciplinas e projetos.
\end{enumerate}

% \section{Pontos críticos para o sucesso da obra}
% 
% A qualidade e adesão desta coleção pelos professores dependerá principalmente de dois pontos: o comprometimento da equipe organizadora com a filosofia da coleção e do alcance que ela terá entre os professores no momento de sua elaboração.

 \section{A equipe}\label{equipe}
 
A equipe de coordenadores do projeto será composta por comitês de elaboração e um comitê de revisão.
Os comitês de elaboração deverão conter ao menos um professor dos anos finais do Ensino Fundamental (EF2). 
Estes grupos podem contar com a colaboração de estudantes que trabalhem com os professores envolvidos. 
Os participantes são especialmente encorajados a desenvolverem monografias de graduação e TCC do Profmat no contexto deste projeto junto com seus estudantes. %, embora não seja recomendada a participação formal no comitê de redação de estudantes do Profmat para que se evite o atraso no cronograma.
O comitê de revisão será formado por escritores renomados e, ao menos dois, professores experientes no EF2.

\clearpage
\begin{table}[h] 
\begin{center}
\begin{tabular}{ll}
PROFESSORES UNIVERSTÁRIOS & PROFESSORES DA EDUCAÇÃO BÁSICA  \\
Fabio Simas - UNIRIO (elaborador) & Leo Akio - CAp UFRJ (revisor)\\
Gladson Antunes - UNIRIO (elaborador) & Felipe Ferreira - SESC e Colégio Santo Ignácio (revisor)\\
Michel Cambrainha - UNIRIO (elaborador)& Eduardo Wagner - FGV (revisor) \\
Victor Giraldo - UFRJ (elaborador)& Luiz Felipe Lins - Prefeitura do Rio de Janeiro (revisor)\\
Humberto Bortolossi - UFF (elaborador)& Letícia Guimarães Rangel - CAp da UFRJ (elaboradora)\\
Wanderley Resende - UFF (elaborador) & gordinho do CP2 que estava no Simpósio em Brasília\\
Estatística e Probabilidade (elaborador) & Sérginho - CP2 (revisor)
\end{tabular}
\end{center}
\end{table}



\noindent{\it Responsabilidades de cada membro na equipe.}

 \begin{enumerate}[A.]
  \item Elaborador
  \begin{enumerate}[(1)]
   \item Escrever material para os livros.
   \item Buscar material que julgue de qualidade, contactar os autores para apresentação do projeto e solicitar atribuição de licença CC BY 4.0. Efetuar as adaptações necessárias antes de submeter à equipe de revisão.
   \item Orientar estudantes em nível de graduação e mestrado em monografias que podem compor trechos dos livros.
  \end{enumerate}

  \item Revisor:
  \begin{enumerate}[(1)]
  \item Analisar a relevância para a coleção do texto submetido.
  \item Analisar a consistência matemática do texto submetido.
  \item Verificar se as habilidades desenvolvidas contemplam aquelas da BNC.
  \item Verificar se os pré-requisitos para o conteúdo já foram trabalhados.
  \item Corrigir o uso da norma culta.
  \item Efetuar a adequação da linguagem para o Ensino Fundamental (quando professor da Educação Básica). 
  \end{enumerate}
  
   \item[B.1.] Revisor de área
   \begin{enumerate}
\item[(8)] Verificar se há saltos no nível de dificuldade do material daquela área.
   \end{enumerate}
   
  \item[B.2.] Revisor de ano letivo
  \begin{enumerate}
\item[(8)] Verificar se há saltos no nível de dificuldade no conteúdo daquele ano.
\item[(9)] Avaliar se a extensão do texto supera o limite estipulado para aquele ano. 
  \end{enumerate}
  \end{enumerate}



\section{Método}
\label{metodo}

O projeto será apresentado em um site na internet, onde poderão ser visualizados e baixados o sumário de cada livro e os capítulos da coleção em suas versões mais recentes.
Haverá um espaço para comentários sobre o conteúdo de cada capítulo.
O usuário poderá editar o texto diretamente, assim como na Wikipédia.
Para isso deverá se registrar e concordar com os termos da licença.
Então com um clique acessará um editor de texto com o código em {\it latex} do capítulo que está visualizando.
Ao término da edição, o revisor é informado das alterações linha por linha.
Estas modificações somente serão incorporadas ao texto disponível para visualização após a concordância do primeiro revisor.

O aparato tecnológico necessário já existe, falta apenas a junção de ferramentas para que a cooperação seja o mais simples possível.
Na seção do site em que figura a visualização de dado arquivo, haverá link para o código-fonte do arquivo no {\it GitHub} (servidor de arquivos com serviço de compartilhamento e controle de versões já utilizado para projetos como este, por exemplo, neste servidor é desenvolvido o Linux).
O {\it GitHub} informa as alterações feitas pelo usuário aos revisores.

O funcionamento destes recursos bem como a aderência dos comitês de elaboração e revisão à esta tecnologia será testada e poderá sofrer alterações antes da divulgação ao grande público.

Serão realizados encontros mensais com quatro horas de duração com os seguintes objetivos principais:
\begin{itemize}
 
 \item Escolha de quais dos ``Objetivos Gerais da Área de Matemática no EF2'' (ver p. \pageref{objetivosBNC}) deve ser atingida nos diversos tópicos que serão abordados (este item pode ser discutido em separado pelos comitês de elaboração com um revisor e depois apresentado aos demais).
 \item Discussão metodológica para abordagem de cada um dos tópicos, como obter inter-relação com as outras 4 áreas da matemática para este nível de escolaridade e outras áreas do conhecimento.
 \item Definição do número de páginas por área no respectivo ano.
 \item Apresentação do andamento dos trabalhos dos comitês de elaboração.
 \item Unificação da linguagem do texto.
 \item Colaboração de todas as equipes com o trabalho de cada comitê de elaboração. 
 \item Aproximação de revisores e elaboradores.  
\end{itemize}

O conteúdo desta obra será aquele determinado pela Base Nacional Curricular (BNC) complementado com alguns tópicos extras a serem selecionados.
A produção inicial ficará a cargo de TCC de graduação e do Profmat sob a orientação dos professores envolvidos, também serão incorporados alguns trabalhos já existentes de autores que aceitem disponibilizá-los nos termos da licença desta coleção.
Além disso, qualquer pessoa que tenha interesse, pode editar o texto, modificando, excluindo ou adicionando novos trechos.

A execução deste projeto se dará em 10 fases:

\begin{enumerate}
\item[Fase 1 -] {\it Estruturação do ambiente virtual de desenvolvimento e organização das equipes.}

Este é o momento para desenvolver o aparato tecnológico necessário ao projeto, recrutar e treinar a equipe para o uso destes recursos. 
Aqui serão definidas preliminarmente as funções de cada membro da equipe e divisão de áreas entre os comitês de elaboração. 
Esta fase termina quando cada participante souber suas atribuições (preliminares), estiver capacitado para utilizar a tecnologia necessária à sua participação e os comitês de elaboração já tiverem definidos as suas metas de habilidades do BNC para o volume do sexto ano.

\item[Fase 2 -] {\it Consolidação do ambiente virtual de desenvolvimento e redação do volume do sexto ano.}

Nesta etapa os comitês de elaboração concluem os volume do sexto ano. Entende-se por ``volume'' o livro do estudante e o respectivo manual do professor.
Os comitês de elaboração manterão seu progresso atualizado no site, permitindo deste modo que os revisores acompanhem e comentem seu desenvolvimento semanalmente.
Somente até esta fase a interface do recurso computacional para controle de versões e colaboração poderá ser modificada consideravelmente. 
Daqui para frente apenas alterações suaves na interface devem acontecer.
Por isso serão convidados alguns professores para colaborar pelo site e apresentar suas impressões sobre o uso da tecnologia. 
A conclusão desta fase será marcada pela finalização do texto do volume do sexto ano e da finalização do aparato tecnológico.

\item[Fase 3 -] {\it Divulgação do material, experimentação e revisão do volume do sexto ano.}

Este é o momento em que o projeto alcança grande visibilidade entre os professores de matemática. Serão utilizados sites de organizações consagradas para a divulgação do material (e.g. OBM, OBMEP, SBM, Profmat, CDME, SBEM, SBE, UFRJ, UFF, UNIRIO). 
Desse modo espera-se obter diversos colaboradores para testar, revisar, opinar e editar o volume do sexto ano. 
A equipe organizadora solicitará aos estudantes do Profmat que experimentem as atividades com seus estudantes do sexto ano e que colaborem compartilhando suas impressões sobre as partes utilizadas do volume.
%A pertinência dos comentários será avaliada pelo comitê de revisores.
Após esta fase o comitê revisor ainda avaliará sugestões de alteração neste volume. 
Contudo esta etapa apenas terminará após este comitê decidir que existe uma primeira versão final para o volume 1 desta obra.

%\item[Fases 4, 6 e 8 -] {\it Redação dos volumes do sétimo, oitavo e nono ano, respectivamente.}

%\item[Fases 5, 7 e 9 -] {\it Experimentação e revisão dos volumes do sétimo, oitavo e nono ano, respectivamente.}

\item[Fases 4 -] {\it Redação do volume do sétimo ano.}

\item[Fases 5 -] {\it Experimentação e revisão do volume do sétimo ano.}

\item[Fases 6 -] {\it Redação do volume do oitavo ano.}

\item[Fases 7 -] {\it Experimentação e revisão do volume do oitavo ano.}

\item[Fases 8 -] {\it Redação do volume do nono ano.}

\item[Fases 9 -] {\it Experimentação e revisão dos volume do nono ano.}

\item[Fase 10 -] {\it Edição da coleção.}
\end{enumerate}
  







%\subsection{Da experimentação do material}

%\subsection{Dos direitos autorais e da licença}

%{\it Creative Commons 4.0} %\url{http://creativecommons.org/licenses/by/4.0/deed.pt_BR}

%\subsection{Da editoração}

%Todo o trabalho de diagramação, impressão e editoração será realizado pelo IMPA.

\clearpage
\section{Cronograma de execução}

As tabelas a seguir descrevem o cronograma de execução das fases do projeto em cada em dos dois anos:
\begin{table}[ht] 
\begin{center}
\begin{tabular}{ccccccccccccc}
2016 & Jan & Fev & Mar & Abr & Mai & Jun & Jul & Ago & Set & Out & Nov & Dez \\
	\hline
Fase 1  & $\bullet$ &  $\bullet$ & & &  \\
\hline
Fase 2 & & & $\bullet$ & $\bullet$ &  & &  \\
\hline
Fase 3 & & & & & $\bullet$ & $\bullet$ & $\bullet$ &\\
\hline
Fase 4 & & & & & & & & $\bullet$ & $\bullet$  & &\\
\hline
Fase 5& & & & & & &  &  &  &$\bullet$ & $\bullet$ &\\
\hline
Fase 6& & & & &  &  &  &  &  &  & & $\bullet$ \\
\hline
\end{tabular}
  \end{center}
  \end{table}

\begin{table}[ht] 
\begin{center}
\begin{tabular}{ccccccccccccc}
2017 & Jan & Fev & Mar & Abr & Mai & Jun & Jul & Ago & Set & Out & Nov & Dez \\
	\hline
Fase 6  & $\bullet$ & $\bullet$ &  & & &  \\
\hline
Fase 7 & & & $\bullet$ & $\bullet$ &  & &  \\
\hline
Fase 8 & & & & & $\bullet$ & $\bullet$ &\\
\hline
Fase 9 & & & & & & & $\bullet$ & $\bullet$ & $\bullet$ &\\
\hline
Fase 10& & & & & &  &  &  &$\bullet$ & $\bullet$ & $\bullet$ & $\bullet$ \\
\hline
\end{tabular}
\end{center}
  \end{table}

  
\section{Bibliografia relacionada ao projeto}

\begin{center}
  Rio de Janeiro, 11 de janeiro de 2016 
\end{center}

% estilo da bibliografia
\bibliographystyle{apalike}
% chamando o arquivo refs.bib
% \bibliography{refs}

\end{document}


 
 
